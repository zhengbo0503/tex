\section{Background Information}
\label{sec.backgr-inform}

In this section, we will provide a brief introduction on the matrix sign
function, the matrix square roots and the polar decomposition. Especially
on the iterations that associated with them. For more comprehensive
analysis and results, see~\ycite[2008, Chap.~6 and 8]{high08_fm}. 

In order to analyze the matrix square roots and the polar decomposition, we
are need the singular value decomposition.

\begin{theorem}[Singular value decomposition]
If $A\in\C\mn$, $m\geq n$, then there exists two unitary matrices
$U\in\C^{m\times m}$ and $V \in \C\nn$ such that
\begin{equation}\notag
  A = U\varSigma V\ctp, \quad \varSigma =
  \diag(\sigma_{1},\dots,\sigma_{n})\in\R^{m\times n},
\end{equation}
where $\sigma_{1},\dots,\sigma_{n}$ are all non-negative and arranged in
non-ascending order. We denote the above decomposition as the singular
value decomposition (we will abbreviate this as SVD) of $A$ and
$\sigma_{1},\dots,\sigma_{n}$ are the singular values of $A$.
\end{theorem}

The number of nonzero singular values of $A$ is the rank of $A$. However,
the number of eigenvalues can be less than the rank of a matrix. For
example, consider $ A =
\begin{bsmallmatrix}
  0 & 2 \\ 0 & 0
\end{bsmallmatrix}
$, then $\Lambda(A) = \{0,0\}$ and $\Sigma(A) = \{2,0\}$, whereas
$\rank(A) = 1$.

The singular values of $A$ is uniquely determined by the eigenvalues of
$A\ctp A$, so the diagonal factor $\varSigma$ is unique up to permutation
of its diagonal entries.


\subsection{Matrix Sign Function}
Before introducing matrix square root and the polar decomposition, we need
the matrix sign function. In this section, we will give a brief
introduction on the matrix sign function. Several iterations for computing
it will be presented. We will see, in Section~\ref{sec.matrix-square-roots}
and Section~\ref{sec.polar-decomposition}, the matrix sign function plays
an important role.

\subsubsection{Definitions and Properties}
The matrix sign function is an analogue of the scalar sign function defined
for $z\in\C$
\begin{equation}\notag
  \sign(z) =
  \begin{cases}
    1, & \Re z > 0, \\ -1, & \Re z < 0.
  \end{cases}
\end{equation}
Notice that $\sign(z)$ is undefined when $z$ lies on the imaginary axis.
Therefore, throughout this section, $A\in\C\nn$ is assumed to have no
eigenvalues on the imaginary axis, so that $\sign(A)$ is defined. Notice
that, this condition restrict $A$ to be nonsingular.

If $A = ZJZ\inv$ is a Jordan canonical form arranged such that
$J = \diag(J_{1},J_{2})$, where the eigenvalues of $J_{1}\in\C^{p\times p}$
lie in the open left half-plane and those of $J_{2} \in \C^{q\times q}$ lie
in the open right half-plane, then
\begin{equation}\notag
  \sign(A) = Z
  \begin{bmatrix}
    -I_{p} & 0 \\ 0 & I_{q}
  \end{bmatrix}
  Z\inv. 
\end{equation}
Clearly, $A$ is commute with $\sign(A)$.

% matrix sign function, explicitly expressed by high94 sec2.
To get another expression of $\sign(A)$, consider the decomposition
$A = SN$, where $S = \sign(A)$. Then
$N^{2} = S\inv A S\inv A = (SS)\inv(A^{2}) = A^{2}$. We denote this
representation as the matrix sign decomposition~\ycite[1994,
Sec.~2]{high94},
\begin{equation}\notag
  A = SN, \quad S = \sign(A),\quad N = (A^{2})^{1/2}.
\end{equation}
Hence, we have the concise formula
\begin{equation}\notag
  \sign(A) = A(A^{2})^{-1/2},
\end{equation}
where $(\cdot)^{1/2}$ denotes the principal square root, see
Theorem~\ref{thm.principal-sqrt}.
% A(A*A)**(-1/2)

% Thm sign(0,A;-A,0) --> ? AB and BA have the same eigenvalue. \cite{Thm.
% 1.3.22}[hojo13\_ma].
The next result is useful for connecting the matrix sign function with the
matrix square root and the polar decomposition.

\begin{theorem}[{\ycite[2005, Lemma~4.3]{hmm05}}] \label{thm.coupled-sign}
Let $A,B\in\C\nn$ and suppose that $AB$ has no eigenvalues on $\R^{-}$.
Then
\begin{equation}\notag
  \sign \left(
  \begin{bmatrix}
    0 & A \\ B & 0
  \end{bmatrix}
  \right) =
  \begin{bmatrix}
    0 & C \\ C\inv & 0 
  \end{bmatrix},
\end{equation}
where $C = A(BA)^{-1/2}$.
\end{theorem}

\begin{proof}
It is obvious that, $AB$ share the same eigenvalues as $BA$~\ycite[2013,
Thm.~1.3.22]{hojo13_ma}. Then consider the matrix
$P = \begin{bsmallmatrix} 0 & A \\ B & 0 \end{bsmallmatrix}$, and this
matrix certainly will not have eigenvalues on the imaginary axis, otherwise
$P^{2} = \begin{bsmallmatrix} AB & 0 \\ 0 & BA \end{bsmallmatrix}$ will
have eigenvalues on $\R^{-}$. Therefore $\sign(P)$ is defined and using the
definition $\sign(A) = A(A^{2})^{-1/2}$, we have
\begin{align}\notag
  \sign \left(
  \begin{bmatrix}
    0 & A \\ B & 0
  \end{bmatrix}
  \right)
  & =
    \begin{bmatrix}
      0 & A \\ B & 0
    \end{bmatrix}
    \left(
    \begin{bmatrix} AB & 0 \\ 0 & BA \end{bmatrix} 
    \right)^{-1/2}\\
      & =
        \begin{bmatrix}
          0 & A(BA)^{-1/2} \\ B(AB)^{-1/2} & 0
        \end{bmatrix}
        =:
        \begin{bmatrix}
          0 & C \\ D & 0
        \end{bmatrix}
\end{align}
Moreover, we impose the fact that $(\sign(P))^{2} = I$, gives $D = C\inv$.
\end{proof}

\subsubsection{Newton's Method}
The most widely used iteration for computing the sign function is the
Newton iteration, which first mentioned by J. D. Roberts~\ycite[1980,
Sec.~1.3]{robe80} when he was solving the matrix Lyapunov equation and the
algebraic Riccati equation. The Newton iteration for the Sign function can
be obtained by applying the first order approximation to the function
$F(X) = X^{2} - I$. Let $Y$ be an approximation to the solution
$X^{2} = I$, then define the correction term $E = S - Y$, where
$S = \sign(A)$. Then $I = S^{2} = (Y+E)^{2} = Y^{2} + EY + YE + E^{2}$.
Since the Newton's method is a first order approximation method, therefore,
we drop the second order term in $E$ and give $I = Y^{2} + EY + YE$.
Assembling these, we have the Newton iteration for the matrix sign
function,
\begin{equation}
  \label{eq.sign-full-newton}
  \left.
  \begin{aligned}
    E_{k}Y_{k} + Y_{k}E_{k} & = I - Y_{k}^{2} \   \\
    Y_{k+1} & = Y_{k} + E_{k} 
  \end{aligned}
  \right\}
  \quad k = 0,1,2,\dots, \quad Y_{0} \text{ is given.}
\end{equation}
However, this method does not have practical use, since we need to solve a
Sylvester equation during each iteration which usually costs $O(n^{3})$
where $n$ is the dimension of $Y_{k}$~\ycite[1979, Sec.~III]{gnv79}.
Instead, we point out that $E_{k} = \frac{1}{2}(Y_{k}\inv - Y_{k})$ is a
solution to the Sylvester equation
$E_{k}^{}Y_{k}^{} + Y_{k}^{}E_{k}^{} = I - Y_{k}^{2}$. Hence we have the
Newton iteration for the matrix sign function
\begin{equation}\label{eq.sign-reduced-newton}
  X_{k+1}^{} = \frac{1}{2}(X_{k}^{} + X_{k}\inv), \qquad X_{0} = A. 
\end{equation}

The next theorem~\ycite[2008,~Thm.~5.6]{high08_fm} gives the convergence of
the above iteration
\begin{theorem}[convergence of the Newton iteration for matrix sign
function]
\label{thm.conv-new-sign}
Let $A\in\C\nn$ have no pure imaginary eigenvalues. Then the Newton
iterates $X_{k}$ in \eqref{eq.sign-reduced-newton} converge quadratically
to $S = \sign(A)$, with
\begin{equation}\notag
  \norm{X^{}_{k+1} - S_{}^{}} \leq \frac{1}{2}
  \norm{X_{k}\inv}\norm{X_{k}^{}-S^{}_{}}^{2} 
\end{equation}
for any consistent norm.
\end{theorem}

\begin{proof}
% the iterates are well defined in two sense 1. lambda remain same
% half-plane, hence sign wouldn't change, defined and nonsingular.
Consider an complex eigenvalue $\lambda = x + \im y$, then the iteration
mapping will map this eigenvalue to
\begin{equation}\notag
  \lambda' = \lambda + \frac{1}{\lambda} =
  x + \im y + \frac{1}{x + \im y} =
  \left( x + \frac{x}{x^{2} + y^{2}} \right) + \im
  \left( y - \frac{y}{x^{2} + y^{2}} \right).
\end{equation}
The mapped eigenvalue $\lambda'$ will remain in the same half plane as
$\lambda$ and never become a pure imaginary eigenvalue as long as $\lambda$
is not pure imaginary. Therefore, all $X_{k}$ are well-defined and
nonsingular.

% thm 1.13 high08\_fm
Since both $X_{k}$ and $S$ are functions of $A$, $X_{k}$ and $S$ are
commute by \ycite[2008, Thm.~1.13]{high08_fm}. Furthermore, we can consider
the difference
\begin{align*}
  X_{k + 1} - S & = \frac{1}{2}(X_{k} + X_{k}\inv - 2S) \\
                & = \frac{1}{2}X_{k}\inv(X_{k}^{2} - 2X_{k}S + I) \\
                & = \frac{1}{2} X_{k}\inv (X_{k} - S)^{2} \quad \text{by
                  commutativity.}                  
\end{align*}
By taking any consistent norm and using $\norm{AB}\le \norm{A}\norm{B}$, we
have the desired norm inequality and this inequality displays the quadratic
convergence.
\end{proof}

\subsubsection{Other Iterations}
One straightforward variant of the Newton
iteration~\eqref{eq.sign-reduced-newton} is the Newton--Schulz iteration.
By considering removing the inversion in the Newton iteration, one can
utilize the one-step Schulz iteration~\ycite[1933]{schu33},
$X_{k+1} = X_{k}(2I - AX_{k})$, to estimate the inversion each step. The
benefit of using Schulz iteration is that, as we iterates $X_k$ from $A$
towards $\sign(A)$, $X_{k}$ is an increasingly good approximation to its
inverse. Hence, to estimate $X_{k}\inv$, we use
$X_{k}\inv \approx X_{k}(2I - X_{k}X_{k})$, and we got the Newton--Schulz
iteration for the matrix sign function
\begin{equation}\notag
  X_{k + 1} = \frac{1}{2}X_{k}(3I - X_{k}^{2}),\quad X_{0} = A.
\end{equation}
This iteration is multiplication-rich and keeps the quadratic convergence
of Newton's method. However, it is only locally convergent for
$\norm{I - A^{2}} < 1$. The convergence result for the Newton--Schulz
iteration was analyzed in terms of Pad\'e iteration by Kenney and
Laub~\ycite[1991, Thm.~5.3]{kela91}. \comment{Any other convergence proof
  except the Pad\'e approximation?}

This ``approximate matrix inverse using Schulz iteration'' approach was
first introduced by Schreiber and Parlett~\ycite[1988, Sec.~3.2.1]{scpa88}.
Higham and Schreiber~\cite[1990]{hisc90} employed just one Schulz iteration
for the polar decomposition and shown its convergence. Later, Kenney and
Laub~\ycite[1991]{kela91} shown that the Newton--Schulz is actually a
Pad\'{e} approximant.

% initial proof of Newton Schulz is obtained by Kenney and Laub in [thm
% 5.3]{kela91}
For non-pure imaginary $z\in\C$, the $\sign(z)$ can be written as
\begin{equation}\notag
  \sign(z) = \frac{z}{(z^{2})^{1/2}} = \frac{z}{(1-(1-z^{2}))^{1/2}} =
  \frac{z}{(1-\xi)^{1/2}},
\end{equation}
where $\xi = 1-z^{2}$. The task of approximating $\sign(z)$ leads to
approximating $h(\xi) = (1-\xi)^{-1/2}$. The theory of the Pad\'{e}
approximants of $h(\xi)$ is based on results for hypergeometric
functions~\ycite[1975, Chap.~5]{bake75_pade}. Kenny and Laub provided a
family of rational iteration functions which are the $[k/m]$ Pad\'{e}
approximants to $h(\xi)$. These functions are rational functions
$P_{km}/Q_{km}$ where $\deg(P_{km}) = k$, $\deg(Q_{km}) = m$, and
\begin{equation}\notag
  h(\xi) - \frac{P_{km}(\xi)}{Q_{km}(\xi)} = O(\xi^{k + m + 1}).
\end{equation}
It turns out that the $[1/0]$ approximant gives the Newton--Schulz
iteration and the $[1/1]$ approximant gives the Halley's method. Here, we
will not discuss these in full detail, but one may refer to the original
paper~\ycite[1991]{kela91} or the book by Higham~\ycite[2008,
Chap.~5.4]{high08_fm}.

\subsubsection{Stability of Sign Iterations}
For $A\in\C\nn$ that has no pure imaginary eigenvalues, by \ycite[2008,
Thm.~5.13]{high08_fm}, if the iteration function $X_{k + 1} = g(X_{k})$ is
superlinearly convergent to $\sign(X_{0})$ for all $X_{0}$ sufficient close
to $\sign(A)$, and $g$ is independent of $X_{0}$, then the iteration is
stable in the sense that the Fr\'echet derivative of the matrix sign
function at $\sign(A)$ has bounded powers.

However, the Newton iteration \eqref{eq.sign-reduced-newton} can be
numerically unstable is not always bounded by a modest multiple of the
condition number $\kappa_{\sign}(A)$\footnote{The condition number is
  defined via \ycite[2008, p.~110]{high08_fm}}, and this is shown by
applying the Newton iteration to a matrix with eigenvalues close to the
imaginary axis~\ycite[2008, pp.~126-127]{high08_fm}. This instability was
also pointed by Bai and Demmel~\ycite[1998, Sec.~2]{bade98}. They observed
that when the eigenvalues of $A$ close to the pure-imaginary axis, the
Newton iteration and its variations are very slowly convergent and may be
mis-convergent.

We can conduct the similar test from \ycite[2008, pp.~126-127]{high08_fm}
\begin{lstlisting}
clc; clear; close all; rng(1);
d = 1/3;
Q = rand_or(16); % generate random orth. mtx using qmult
T = triu(randn(16)); T_diag = abs(diag(T));
T_diag(1:8) = d * T_diag(1:8); 
T_diag(9:16) = -1 * d * T_diag(9:16);
T = T - diag(diag(T)) + diag(T_diag);
A = Q * T * Q';
[X,k] = signm_newton(A);
fun = @(x) x/abs(x); 
k_sign = funm_condest_fro(A,fun); % estimate the condition number
\end{lstlisting}
The functions \inline{signm_newton} at line 9 and \inline{funm_condest_fro}
at line 11 are from \cite{high-mft}. With $d = 1/3$, the approximate
condition number is $\kappa_{\sign} \approx \text{2.1e8}$. However,
$\min_{k}\inorm{S - X_{k}}/\inorm{S} \approx \text{7.8e-2}$, which greatly
exceeds $\kappa_{\sign}u \approx \text{2.4e-8}$.

\subsection{Matrix Square Roots}
\label{sec.matrix-square-roots}

We will concerned with the principal square root \ycite[2008,
Thm.~1.29]{high08_fm}. We denote by $\R^{-}$ the closed negative real axis.

\begin{theorem}[principal square root]
\label{thm.principal-sqrt}
Let $A\in\C\nn$ have no eigenvalues on $\R^{-}$. There is a unique square
root $X$ of $A$ all of whose eigenvalues lie in the open right half-plane,
and it is a primary matrix function of $A$. We refer to $X$ as the
principal square root of $A$ and write $X = A^{1/2}$.
\end{theorem}


\subsubsection{Schur Method}
\label{sec.schur-method}
Schur method, which initially presented by Bj\"{o}rck and Hammarling
\ycite[1983]{bjha83} and extended to compute a real square root of a real
matrix in real arithmetic by Higham~\ycite[1987]{high87_rsqrt}, can
computes all the primary square root of a given matrix provided its
eigenvalues does not lie on $\R^{-}$.

\begin{algorithm}[H]
\caption{(Schur method) Given a nonsingular $A\in\C\nn$ this algorithm
  computes $X = \sqrt{A}$ via a Schur decomposition, where $\sqrt{\cdot}$
  denotes any primary square root.}
\label{alg.schur}
\begin{algorithmic}[1]
\State{Compute a (complex) Schur decomposition $A = QTQ\ctp$.}
\State{$u_{ii} = \sqrt{t_{ii}},\ i = 1\colon n$}
\For{$j = 2\colon n$}
\For{$i = j - 1\colon -1 \colon 1$}
\State{$\displaystyle u_{ij} = \frac{t_{ij} -
    \sum_{k=i+1}^{j-1}u_{ik}u_{kj}}{u_{ii} + u_{jj}}$}
\EndFor
\EndFor
\State{$X = QUQ\ctp$}
\end{algorithmic}
\end{algorithm}
The cost of this algorithm is $25n^{3}$ for the Schur
decomposition~\ycite[2013, Alg.~7.5.2]{gova13_mc4} plus $n^{3}/3$ for $U$
and $3n^{3}$ to form $X$, which result in $28 \frac{1}{3}n^{3}$ flops in
total.

Following the language of \ycite[2002, Sec.~3.1]{high02_asna2}, if the
computed diagonal elements satisfy
$\wh u_{ii} = \sqrt{t_{ii}}(1 + \delta_{i})$, where
$\abs{\delta_{i}} \le u$. Then, we have
\begin{equation}\notag
  \wh U^{2} = T + \varDelta T, \qquad \abs{\varDelta T} \le
  \wt \gamma_{n}\abs{\wh U}^{2},
\end{equation}
where the inequality is to be interpreted elementwise. Computation of the
Schur decomposition by the QR algorithm is backward stable process
\ycite[2013, Sec.~7.5.6]{gova13_mc4}, and we have
\begin{equation}\notag
  \wh X^{2} = A + \varDelta A, \qquad \fnorm{\varDelta A} \le \wt
  \gamma_{n^{3}} \fnorm{\wh X}^{2}.
\end{equation}
Using the expression $\alpha_{F}(\wh X) = \fnorm{\wh X}^{2}/\fnorm{A}$, we
can rewrite the expression as
\begin{equation}\notag
  \frac{\fnorm{A - \wh X^{2}}}{\fnorm{A}} \le \wt
  \gamma_{n^{3}}\alpha_{F}(\wh X).
\end{equation}
This analysis confirms the result in \ycite[1983]{bjha83}, which also shown
that if $\alpha$ is not large, then the computed square root is the exact
square root of a matrix close to $A$.

A summary of the extension of the Schur method by
Higham~\ycite[1987, Sec.~4.3]{high87_rsqrt} can be found in
\ycite[2008, Sec.~6.2]{high08_fm}. Later in 2013, Deadman, Higham and
Ralha~\ycite[2013]{dhr13} provided blocked Schur algorithms. MATLAB
currently uses the Schur method with recursive blocking~\ycite[2020,
Sec.~4]{hiho20} described in~\ycite[2013]{dhr13}. 

Notice that, for Hermitian matrices, the Schur decomposition becomes the
eigendecomposition, and the Schur method becomes
Algorithm~\ref{alg:sqrt-eigdecomp}.



\subsubsection{Newton's Method}














\subsection{Polar Decomposition}
\label{sec.polar-decomposition}

\begin{theorem}[Polar decomposition]
Let $A\in\C\nn$ be nonsingular. There exist a unique unitary matrix
$U\in\C\nn$, and a unique Hermitian positive definite matrix $H\in\C\nn$
such that $A = UH$. The matrix $H$ is determined by
\begin{equation}\notag
  (A\ctp A)^{1/2} = (H\ctp U\ctp U H)^{1/2} = (H^{2})^{1/2} = H. 
\end{equation}
Moreover, $U$ is uniquely determined by $U = AH\inv$.
\end{theorem}

\begin{proof}
Full proof for general $A\in\C\mn$ can be found in \ycite[2008,
Thm~8.1]{high08_fm}. Let $A\in\C\nn$ be nonsingular, and
$A = P\varSigma Q\ctp$. Then, $H = Q\varSigma Q\ctp$ and $U = PQ\ctp$. We
can check $H$ and $U$ by
\begin{align*}
  H^{2} & = Q\varSigma\varSigma Q\ctp = Q\varSigma P\ctp P\varSigma Q =
          (P\varSigma Q\ctp)\ctp (P\varSigma Q) = A\ctp A, \\
  U & = AH\inv = P\varSigma Q\ctp Q\varSigma\inv Q\ctp = PQ\ctp.
\end{align*}
The uniqueness of $H$ is ensured by \ycite[2008, Cor.~1.30]{high08_fm}.
\end{proof}

We will refer to $U$ as the unitary polar factor and $H$ as the Hermitian
polar factor. It is sufficient to only consider the square, nonsingular
matrices for the polar decomposition. Let $A\in\C\mn$ with $m\geq n$ have
the QR factorization $A = QR$ where $Q\in\C^{m\times n}$ and $R\in\C\nn$.
Then find the polar decomposition $R = UH$ and $A = QU\cdot H$ becomes the
polar decomposition of $A$. In addition, if $A\in\C\mn$ is singular, we can
compute a complete orthogonal decomposition

\begin{equation}\notag
  A = P
  \begin{bmatrix}
    R & 0 \\ 0 & 0 
  \end{bmatrix}
  Q\ctp,
\end{equation}
where $P,Q$ are unitary, and $R\in\C^{r\times r}$ is nonsingular and upper
triangular. For more details of this decomposition, you may refer to
Bj\"{o}rck~\ycite[1996, Def.~1.3.1]{bjor96_nmflsp}. Then, we may compute
the polar decomposition of $R$ and assemble them together, according to
\ycite[1990, Sec.~2]{hisc90},
\begin{align*}
  A & = P
      \begin{bmatrix}
        UH & 0 \\ 0 & 0
      \end{bmatrix}
      Q\ctp \\
    & = P
      \begin{bmatrix}
        U & 0 \\ 0 & I_{m-r,n-r}
      \end{bmatrix}
      \begin{bmatrix}
        H & 0 \\ 0 & 0
      \end{bmatrix} Q\ctp \\
    & = P
      \begin{bmatrix}
        U & 0 \\ 0 & I_{m-r,n-r}
      \end{bmatrix}
      Q\ctp Q
      \begin{bmatrix}
        H & 0 \\ 0 & 0
      \end{bmatrix} Q\ctp \\
    & \equiv U_{A}H_{A},
\end{align*}
where $I_{m-r,n-r}$ is the $(m-r)\times (n-r)$ identity
matrix.\footnote{The identity matrix is $I_{m,n} = (\delta_{ij})\in\R\mn$.
  \ycite[2008, App.~B.1]{high08_fm}}


The fundamental connection between the matrix sign function and the polar
decomposition is obtained from Theorem~\ref{thm.coupled-sign} by setting
$B = A\ctp$,
\begin{equation}\notag
  \sign \left(
  \begin{bmatrix}
    0 & A \\ A\ctp & 0
  \end{bmatrix}
  \right) =
  \begin{bmatrix}
    0 & A(A\ctp A)^{-1/2} \\ (A\ctp A)^{1/2}A\inv & 0
  \end{bmatrix}
  =
  \begin{bmatrix}
    0 & U \\ U\ctp & 0
  \end{bmatrix}.
\end{equation}







% convergence for the polar-newton-schulz is discussed in bjbo71 --> 0 <
% sigma(A) < sqrt(3)

%%% Local Variables:
%%% mode: latex
%%% TeX-master: "do"
%%% End:
