\documentclass{article}
\def\ntitle {Update 1}
% \def\nauthor{ } % default author is Zhengbo Zhou
% \def\notes{ } % default is the submitted version
% \def\ndate{ } % default is today's date
% \def\needcrop{ } % crop for easy previewing
% \def\fancysec{ } % section font become fancier 
\RequirePackage{etex}
\makeatletter
\ifx \nauthor\undefined
  \def\nauthor{Zhengbo Zhou}
\else 
\fi 
\ifx \ndate\undefined 
  \def\ndate{\today}
\else 
\fi 

\author{\nauthor \thanks{%
Department of Mathematics,
University of Manchester,
Manchester, M13 9PL, England
(\texttt{zhengbo.zhou@postgrad.manchester.ac.uk}).
}}
\date{\ndate}
\title{\ntitle}

% RedeclareMathOperator
\newcommand\RedeclareMathOperator{%
  \@ifstar{\def\rmo@s{m}\rmo@redeclare}{\def\rmo@s{o}\rmo@redeclare}%
}
% this is taken from \renew@command
\newcommand\rmo@redeclare[2]{%
  \begingroup \escapechar\m@ne\xdef\@gtempa{{\string#1}}\endgroup
  \expandafter\@ifundefined\@gtempa
     {\@latex@error{\noexpand#1undefined}\@ehc}%
     \relax
  \expandafter\rmo@declmathop\rmo@s{#1}{#2}}
% This is just \@declmathop without \@ifdefinable
\newcommand\rmo@declmathop[3]{%
  \DeclareRobustCommand{#2}{\qopname\newmcodes@#1{#3}}%
}
\@onlypreamble\RedeclareMathOperator

\usepackage{algorithm}
\usepackage{algpseudocode}
\usepackage{comment}
\usepackage{bookmark}
\usepackage{microtype}
\usepackage{booktabs}
\usepackage{lastpage}
\usepackage{fancyhdr}
\usepackage{amsthm}
\usepackage{mathtools}
\usepackage{enumerate}
\usepackage{mathrsfs}
\usepackage{amsfonts}
\usepackage{amssymb}
\usepackage{tcolorbox}
\usepackage{bm}
\usepackage{cancel}
\usepackage{bbm}
\usepackage{accsupp}
\usepackage{enumitem}
\usepackage{hyperref}
\usepackage[fontsize=12pt]{fontsize}
\usepackage{geometry}
\usepackage{microtype}
\usepackage{algorithm,algorithmicx,algpseudocode}

\hbadness=99999

\geometry{
  a4paper,
  textwidth=165truemm,
  textheight=240truemm,
  top=28.5truemm,
}

\let\LaTeXStandardTableOfContents\tableofcontents
\renewcommand{\tableofcontents}{%
\begingroup%
\renewcommand{\bfseries}{\sc}%
\LaTeXStandardTableOfContents%
\endgroup%
}%
\setcounter{tocdepth}{3}
\makeatletter
\renewcommand\tableofcontents{\@starttoc{toc}}
\makeatother

\hypersetup{
  hypertexnames=false, 
  colorlinks=true,
  linkcolor=blue,
  pdfauthor={Zhengbo Zhou},
  pdftitle={\ntitle},
  pdfcreator={Zhengbo Zhou via MikTeX}
}

\usepackage[hyperpageref]{backref}
\renewcommand*{\backref}[1]{}
\renewcommand*{\backrefalt}[4]{
    \ifcase #1 %
    No citations.%
    \or
    (Cited on p.~#2.)
    \else
    (Cited on pp.~#2.)
    \fi
}



\pagestyle{fancyplain}
\fancyhead[R]{{\textit{\footnotesize\nouppercase Short Course : Function of Matrices}}}
\fancyhead[L]{\footnotesize\nouppercase\leftmark}


\newtcolorbox{mybox}[1]{colback=white!90!black,colframe=white!40!black,fonttitle=\scshape\centering,title=#1}

%%% MATLAB Code From Dr. Chris Johnson 
\usepackage{color}  
\usepackage{xcolor}
\usepackage{listings}
\definecolor{codegreen}{rgb}{0,0.6,0}
\definecolor{codegray}{rgb}{0.5,0.5,0.5}
\definecolor{codepurple}{rgb}{0.58,0,0.82}
\definecolor{mygreen}{RGB}{28,172,0}
\definecolor{mylilas}{RGB}{170,55,241}
\definecolor{backcolour}{rgb}{0.95,0.95,1.92}
\lstdefinestyle{mystyle}{
	language=matlab,
    commentstyle=\color{codegreen},
    keywordstyle=\color{blue},
    numberstyle=\footnotesize\color{codegray},
    stringstyle=\color{codepurple},
    basicstyle=\linespread{1}\ttfamily\small,
    breakatwhitespace=false,
    breaklines=true,
    captionpos=b,
    keepspaces=true,
    numbers=left,
    numbersep=10pt,
    showspaces=false,
    showstringspaces=false,
    showtabs=false,
    tabsize=4,
    aboveskip=\medskipamount,
    % frame=single,
}
\lstset{style=mystyle}
\def\inline{\lstinline[basicstyle=\upshape\ttfamily]}

%% Theorems 
\newtheorem{theorem}{Theorem}[section]
\newtheorem{proposition}[theorem]{Proposition}
\newtheorem{corollary}[theorem]{Corollary}
\newtheorem{lemma}[theorem]{Lemma}
\theoremstyle{definition}
\newtheorem{definition}[theorem]{Definition}
\newtheorem{example}[theorem]{Example}
\newtheorem{remark}[theorem]{Remark}
\newtheorem*{question}{Question}
\newtheorem*{recall}{Recall}
\newtheorem*{assumption}{Assumption}
\newtheorem*{note}{Note}
\numberwithin{equation}{section}

\let\stdsection\section
\renewcommand\section{\newpage\stdsection}

%%% Paired labels
\DeclarePairedDelimiter\ceil{\lceil}{\rceil}
\DeclarePairedDelimiter\floor{\lfloor}{\rfloor} 
\DeclarePairedDelimiter\abs{\lvert}{\rvert} % |a|
\DeclarePairedDelimiter\inner{\langle}{\rangle} % <a>
\makeatletter
\let\oldabs\abs
\def\abs{\@ifstar{\oldabs}{\oldabs*}}

%%%------------------------------------------------------------------%%%

%%% vector bold
\renewcommand{\vec}[1]{\bm{#1}}

%%%------------------------------------------------------------------%%%

%%% Real and Imaginary
\RedeclareMathOperator{\Im}{\mathrm{Im}}
\RedeclareMathOperator{\Re}{\mathrm{Re}}

%%% Integrate from ... to ...       
\newcommand{\intii}{\int_{-\infty}^{\infty}}

%%% Greek Letters
% NEVER define \l for \lambda due to Polish names in BibTeX
\renewcommand{\L}{\Lambda}
\newcommand{\vL}{\varLambda}
\newcommand{\g}{\gamma}
\newcommand{\G}{\Gamma}
\newcommand{\vG}{\varGamma}
\renewcommand{\o}{\omega}
\renewcommand{\O}{\Omega}
\newcommand{\vO}{\varOmega}
\newcommand{\s}{\sigma}
\renewcommand{\S}{\Sigma}
\newcommand{\vS}{\varSigma}
\newcommand{\eps}{\varepsilon}
\newcommand{\lap}{\varDelta}

%%% Matrix Related
\newcommand{\n}{^{n}}
\newcommand{\m}{^{m}}
\newcommand{\Rn}{\R^n}
\newcommand{\mn}{^{m\times n}}
\newcommand{\nn}{^{n\times n}}
\newcommand{\tp}{^{T}} 
\newcommand{\ctp}{^{*}}
\newcommand{\inv}{^{-1}}
\DeclareMathOperator{\diag}{diag}
\DeclareMathOperator{\rank}{rank}
\DeclareMathOperator{\tr}{trace}
\DeclareMathOperator{\range}{Range}

%%%------------------------------------------------------------------%%%

%% Norms
\newcommand{\iter}[1]{^{(#1)}} % iteration
\newcommand{\gnorm}[1]{\left\|{#1}\right\|} % general norm
\newcommand{\norm}[1]{\gnorm{#1}}
\newcommand{\tnorm}[1]{\gnorm{#1}_2} % 2-norm
\newcommand{\inorm}[1]{\gnorm{#1}_\infty} % infinity norm

%%% Over the expressions
\newcommand{\wh}{\widehat}
\newcommand{\wt}{\widetilde}
\newcommand{\wb}{\overline}

%%% Calculus
\DeclareMathOperator{\grad}{\nabla}
\renewcommand{\div}{\nabla\cdot}
\DeclareMathOperator{\curl}{\nabla\times}
\newcommand{\dd}{\mathrm{d}}
\newcommand{\pp}{\partial}
\def\eu{\mathrm{e}} % euler's constant
\def\im{\mathrm{i}} % imaginary unit

%%% Citation
\def\ycite[#1#2#3#4#5]#6{\cite[$\mit{#1#2#3#4}$#5]{#6}}

%%%------------------------------------------------------------------%%%

%%% MATHBB
\newcommand{\mb}[1]{\mathbb{#1}}
\newcommand{\N}{\mb{N}}
\newcommand{\Z}{\mb{Z}} 
\newcommand{\Q}{\mb{Q}}
\newcommand{\R}{\mb{R}} 
\newcommand{\C}{\mb{C}}
\newcommand{\F}{\mb{F}}
\renewcommand{\P}{\mb{P}} % Probability
\newcommand{\E}{\mb{E}} % Expectation
\newcommand{\V}{\mb{V}} % Variance

%%% MATHCAL and MATHSCR
\newcommand{\mc}[1]{\mathcal{#1}} % For spaces 
\newcommand{\ms}[1]{\mathscr{#1}} % For sigma-algebra
\newcommand{\mf}[1]{\mathfrak{#1}}

%%%------------------------------------------------------------------%%%

%%% Stochastic Calculus
\newcommand{\ps}{$(\Omega,\ms F,\P)$}
\newcommand{\ito}{It\^o}
\DeclareMathVersion{bold}
\newcommand{\indi}{\mathbbm{1}} % indicator function
\providecommand*{\napprox}{%
  \BeginAccSupp{method=hex,unicode,ActualText=2249}%
  \not\approx
  \EndAccSupp{}%
}
\mathchardef\gang="2D
\DeclareMathOperator*{\esssup}{esssup}

\DeclareMathOperator{\law}{\mathfrak{Law}}
\newcommand{\loc}{\textup{loc}}
\newcommand{\locmart}{\mc M_{\loc}^C}
\newcommand{\locm}{\mc M_{\loc,T}^{C,0}}
\newcommand{\locmt}{\wt{\mc M}_{\loc,T}^{C,0}}
\newcommand{\nicee}[1]{\mathcal{E}_{#1}^H(B)}


\DeclareMathOperator{\sign}{diag}

\makeatother

%% redefine \emph and \bf such that they are colored and can be easily
%% transformed back to normal \emph and \bf
\renewcommand{\emph}[1]{\textit{\color{purple} #1}}
\renewcommand{\bf}[1]{\textsf{\bfseries \color{purple} #1}}

\def\sign{\mathrm{sign}}

\begin{document}
\maketitle
% \tableofcontents
\thispagestyle{firstpage} 
%%%%%%%%%%%%%%%%%%%%%%%%%%%% MAIN ARTICLE %%%%%%%%%%%%%%%%%%%%%%%%%%%%

\section{Postive Definiteness, SVD and Eigendecomposition}
\label{sec:norm-post-defin}

Suppose $A \in\C\nn$, and $\Lambda$ is the diagonal matrix with eigenvalues
of $A$ lies on its diagonal in arbitrary order. Denote $\Lambda(A)$ as all
the eigenvalues of $A$, and $\Sigma(A)$ as all the singular values of $A$.
We use SVD to denote ``Singular Value Decomposition''.

\begin{enumerate}
\item\label{item:1} We know that if $A$ is normal, then it can be unitarily
diagaonlizable, i.e. $A = Q\Lambda Q\ctp$, where $Q$ is a unitary matrix.
\item If $A$ is Hermitian, then its eigenvalues are all real and
$\Sigma(A) = \abs{\Lambda(A)}$. If the SVD of $A$ is $U\Sigma V\ctp$, then
$\abs{U} = \abs{V}$. They are not agreed when the eigenvalues and
``corresponding'' singular values are not agreed.
\item $A$ is Hermitian, then its eigenvalues are all real. (one can easily
prove this by using~\ref{item:1}).
\item \label{item:4} If $A$ is normal, and $\Lambda(A) \subset \R$, then
$A\ctp = A$.
\begin{proof}
Since $A$ is normal, then by~\ref{item:1}, $A$ is unitarily diagonalizable,
$A = Q\Lambda Q\ctp$. Then $A\ctp = Q\Lambda\ctp Q\ctp = Q\Lambda Q\ctp$.
The final equality uses $\Lambda(A) \subset \R$, i.e.
$\wb{\Lambda(A)} = \Lambda(A)$.
\end{proof}
\item The previous two points can be concluded as: \emph{If $A$ is normal,
  then $A$ is Hermitian if and only if its eigenvalues are all real.}
\item Using \ref{item:4}, we conclude that a normal matrix that is not
Hermitian must have complex eigenvalues. E.g.
\begin{equation}\notag
  A = 
  \begin{bmatrix}
    1 & 1 + \im & 1 \\
    -1 + \im & 1 & 1 \\
    -1 & -1  & 1
  \end{bmatrix}
\end{equation}
This matrix is normal, but not Hermitian, and
$\Lambda(A) = \{1+ 2.21\im, 1 - 1.68\im, 1 - 0.54\im\}$. Then I wondering,
why these complex eigenvalues does not comes in conjugate pairs.
\item Complex eigenvalues of matrices with \emph{real} entries come as
conjugate pairs.
\end{enumerate}

\begin{lemma}[SVD and Eigendecomposition] \label{lem:svd-eig} Suppose
$A\in\R\nn$ has a SVD $U\Sigma V\ctp$. Then, we have the following
\begin{align*}
  & AA\ctp = U\Sigma V\ctp V \Sigma\ctp U\ctp = U(\Sigma^2)U\ctp,\\
  & A\ctp A = V\Sigma\ctp U\ctp U\Sigma V\ctp = V(\Sigma^2)V\ctp.
\end{align*}
\begin{enumerate}
\item Hence the singular values $\sigma_1,\dots,\sigma_n$ are the positive
square roots of the eigenvalues of $AA\ctp$ or $A\ctp A$. Moreover, since
$AA\ctp$ and $A\ctp A$ are positive semi-definite, hence the square roots
are real and non-negative.
\item The left singular vectors, columns of $U$ are the eigenvectors of
$AA\ctp$.
\item The right singular vectors, columns of $V$ are the eigenvectors of
$A\ctp A$.
\end{enumerate}
\end{lemma}

\begin{proposition} [{\cite[Theorem~5.5]{trba97_nla}}]
\label{prop:sv=|ev|}
If $A = A\ctp\in \C\nn$, then the singular values of $A$ are the absolute
values of the eigenvalues of $A$.
\end{proposition}

\begin{proof}
This proof is done by construction of SVD. $A$ is Hermitian, hence we have
$A = Q\Lambda Q\ctp$, where $\Lambda\in\R\nn$, and $Q$ is unitary. We can
write the eigendecomposition as

\begin{equation}
  \label{eq:2}
  A = Q\Lambda Q\ctp = Q |\Lambda| \sign(\Lambda) Q\ctp 
  =: Q|\Lambda| P\ctp, 
\end{equation}

where $\abs{\Lambda}$ and $\sign(\Lambda)$ denote the diagonal matrices
whose entries are $\abs{\lambda_i}$ and $\sign(\lambda_i)$, respectively.
Notice that $P$ is unitary by noticing $\sign(\Lambda)$ is unitary.
Therefore, \eqref{eq:2} is a SVD of $A$, with singular values equal to
$\abs{\lambda_i}$. One can easily order the diagonal entries of $|\Lambda|$
by applying suitable permutation matrices to $Q$ and $\sign(\Lambda)Q\ctp$.
\end{proof}

\begin{corollary}
\label{prop:SVD-u=v}
If $A = A\ctp\in \C\nn$, and it has a singular value decomposition
$A = U\Sigma V\ctp$. Then $\abs{U} = \abs{V}$.
\end{corollary}

\begin{proof}
({\cite[Section~3.1]{dmm03}}) Let $A\in\R\nn$ be a symmetric matrix with
SVD $A = U\Sigma V\tp$. Then $V\tp A V = V\tp U \Sigma$, and $V\tp U\Sigma$
is orthogonally similar to $A$, with $V\tp U$ is orthogonal.

If $A$ has distinct singular values
$\sigma_1 > \sigma_2 > \dots > \sigma_p$ with respective multiplicities
$m_i$, $i=1,\dots,p$. ($\sum_i^pm_i=n$). Partitioning $U$ and $V$ into
\begin{equation}\notag
  U = [  U_1 \mid \dots \mid   U_p], \qquad
  V = [  V_1 \mid \dots \mid   V_p]
\end{equation}
where $  U_i,   V_i \in \R^{n\times m_i}$ corresponding to each
distinct singular value. Since, due to the symmetry of $A$ and
Lemma~\ref{lem:svd-eig}, both its left and right singular vectors (columns
of $U$ and $V$) are eigenvectors of $A^2$. Consequently,
\begin{equation}\notag
  V\tp U = \diag(  V_1\tp   U_1,\dots,   V_p\tp   U_p)
\end{equation}
is block diagonal, where each diagonal block
$  V_i\tp U_i\in \R^{m_{i}\times m_{i}}$ is itself \emph{orthogonal}.
Furthermore, since
\begin{equation}\notag
  V\tp A V = V\tp U \Sigma = \diag(\sigma_1   V_1\tp  
  U_1,\dots,\sigma_p   V_p\tp   U_p)
\end{equation}
is symmetric, we conclude that each $  V_i\tp   U_i$ is not only
orthogonal but also symmetric, therefore, its eigenvalues are $\pm 1$. The
$\pm 1$ are precisely the signs of those eigenvalues of $A$ having modulus
$\sigma_{i}$.

\begin{example}
Suppose $A$ has an eigendecomposition
\begin{equation}\notag
  A = \wt V
  \begin{bmatrix}
    4 & & & \\ & -3 & & \\ & & 3 & \\ & & & 5
  \end{bmatrix} \wt V\tp,
\end{equation}
then the SVD of $A$ have the form
\begin{equation}\notag
  A = U
  \begin{bmatrix}
    5 & & & \\ & 4 & & \\ & & 3 & \\ & & & 3
  \end{bmatrix}.
\end{equation}
Since $V\tp U \Sigma$ is orthogonally similar to $A$, therefore,
\begin{equation}\notag
  V\tp U =
  \begin{bmatrix}
    1 & & & \\ & 1 & & \\ & & -1 & \\ & & & 1
  \end{bmatrix}
\end{equation}
\textbf{(END OF EXAMPLE)}
\end{example}

In the simplest case when $m_{i} = 1$, then eigenvalues are just $v_{i}\tp
u_{i}\sigma$. In the general case, a simple calculation shows that if the
spectrum (eigenvalues) of $  V_{i}\tp   U_{i}$ contains $m_{i}^{+}$
eigenvalues equal to 1 and $m_{i}^{-}$ equal to $-1$, ($m_{i} = m_{i}^{+} +
m_{i}^{-}$), then
\begin{equation}\label{eig-trace}
  m_{i}^{\pm} = \frac{m_{i} \pm \tr(  V_{i}\tp   U_{i})}{2},
\end{equation}
i.e. the multiplicity of the eigenvalues $\pm \sigma_{i}$ can be easily
recovered from the trace of $  V_{i}\tp   U_{i}$.

Now, in order to obtain eigenvectors of $A$, we split into the following
three cases:
\begin{enumerate}
\item
If $m_{i} = 1$, then right singular vectors $v_{i}$ itself is an
eigenvector.
\item
If $m_{i} > 1$, but $\tr(  V_{i}\tp   U_{i}) = m_{i}$ (the eigenvalues
of $V_{i}\tp U_{i}\tp$ are all positive 1 by \eqref{eig-trace}), then all
the $m_{i}$ eigenvalues are all equal  to $\sigma_{i}$, and the
eigenvectors are the columns of $  V_{i}$. Analog applies to $\tr( 
V_{i}\tp   U_{i}) = -m_{i}$. 
\item
Generally, if $m_{i}>1$ and $m_{i} \neq m_{i}^{\pm}$, consider for each $i
= 1,\dots,p$, we have an orthogonalization $V_{i}\tp U_{i} =
W_{i}J_{i}W_{i}\tp$, with $J_{i} = \diag(I_{m_{i}^{+}},-I_{m_{i}^{-}})$ and
$W_{i} = [W_{i}^{+}\mid W_{i}^{-}]\in \R^{m_{i}\times m_{i}}$ partitioned
accordingly. Then, denoting $W = [W_{1},\dots,W_{p}]$, we have
\begin{equation}\notag
  V\tp U \Sigma =
  \begin{bmatrix}
    \Sigma_{1}^{+} & & & &  \\
                   & \Sigma_{1}^{-} & & & \\
                   & & \ddots & & \\
                   & & & \Sigma_{p}^{+} & \\
                   & & & & \Sigma_{p}^{-}
  \end{bmatrix} =: \wt \Sigma.
\end{equation}
Therefore, we successfully recover the eigenvalues from the singular values
by using the left and right singular vectors.
Then by the relation $V\tp A V = V\tp U \Sigma$, we have
\begin{equation}\notag
  A = V\wt\Sigma V\tp.
\end{equation}
\end{enumerate}
\end{proof}

For this interesting mathematical analysis, we need to do a numerical test.
\begin{lstlisting}[numbers=none]
>> rng = 1; A = randn(10); A = A + A'; % generate symmetric matrix
>> [U,S,V] = svd(A); D = V'*U*S;
>> disp(norm(A*V - V*D)/norm(A));
    1.1880e-15
\end{lstlisting}
The numerical experiment shows that  $V\tp U \Sigma$ indeed
``apporoximate'' the eigenvalues.  



% \begin{proof}[Proof. Not Correct?]
% From Proposition~\ref{prop:sv=|ev|}, $A$ has a singular value
% decomposition $A = Q |\Lambda| \sign(\Lambda) Q\ctp$. Then
% \begin{equation}\notag
% |U| = |Q| = |Q\sign(\Lambda)\ctp| = |V|.\qedhere
% \end{equation}
% \end{proof}

\begin{corollary} \label{cor:svd=evd} Let $A\in\C\nn$ be Hermitian positive
definite, then its eigendecomposition coincide with its singular value
decomposition.
\end{corollary}

\begin{proof}
By noting if $A$ is Hermitian positive definite, and its eigendecomposition
is $Q\Lambda Q\ctp$, then $\sign(\Lambda) = I$.
\end{proof}

Notice that, previous construction does not care about the order of the
diagonal entries of $\Lambda$, since we can always permute $Q$ and
$\Lambda$ to make $\Lambda$ sorted. This can be done using MATLAB.

\begin{lstlisting}
% https://uk.mathworks.com/help/matlab/ref/eig.html
function [Vs,Ds] = myeig(A)
[V,D] = eig(A);
[~,ind] = sort(diag(abs(D)),'descend');
Ds = D(ind,ind);
Vs = V(:,ind);
end 
\end{lstlisting}
On the 4th line of the code, I use \inline{abs(D)} in order to sort the
eigenvalues with respect to their absolute values, which is the singular
values. We can then check the function via
\begin{lstlisting}[numbers=none]
>> rng(1); A = randn(5); A = A + A'; % symmetric matrix
>> [V,D] = myeig(A); 
>> disp(diag(D)); disp('---'); disp(norm(A*V-V*D));
  -6.7126e+00
   4.9099e+00
  -4.3623e+00
  -1.1499e+00
   5.0084e-01
---
   2.4697e-15
\end{lstlisting}
The function \inline{myeig} successfully create a sorted eigendecomposition
in the sense that the eigenvalues are sorted in the descending order in
modulus. Moreover, to support Proposition~\ref{prop:sv=|ev|}, we check by
\begin{lstlisting}[numbers=none]
>> svd(A) - abs(diag(D))
ans =
   8.8818e-16
   8.8818e-16
   3.5527e-15
   6.6613e-16
  -8.8818e-16
\end{lstlisting}
These numbers are around unit roundoff and considered as zeros.

\textbf{Conclusion.} To compute the matrix principal square roots for
Hermitian positive definite matrix, we aim to compute it via
eigendecomposition, since $A^{1/2} = Q\Lambda^{1/2}Q\ctp$. By
Corollary~\ref{cor:svd=evd}, we can simply compute the singular value
decomposition via DV-SVD algorithm\footnote{This algorithm refer to the SVD
  algorithm proposed by Drma{\v{c}} and Veseli{\'{c}}
  in~\cite{drve08i,drve08ii}}.

%%%%%%%%%%%%%%%%%%%%%%%%%%%%%%%%%%%%%%%%%%%%%%%%%%%%%%%%%%%%%%%%%%%%%%
\section{Path to Matrix Square Root of SPD Matrix}
Normally, we can compute the principal square root of any positive definite
matrix via the following algorithm proposed by Higham in~\ycite[2008,
Algorithm~6.21]{high08_fm}.

\begin{algorithm}[h]
\caption{Given a Hermitian positive definite matrix $A\in \C\nn$ this
  algorithm computes $H = A^{1/2}$.}
\label{alg:sqrt-higham}
\begin{algorithmic}[1]
\State{Compute the Cholesky factorization $A = RR\tp$.}
\State{Compute the Hermitian polar factor $H$ of $R$ by applying any method
  (exploiting the triangularity of $R$).} 
\end{algorithmic}
\end{algorithm}

\begin{algorithm}[h]
\caption{Given a Hermitian positive definite matrix $A\in \C\nn$ this
  algorithm computes $H = A^{1/2}$.}
\label{alg:sqrt-eigdecomp}
\begin{algorithmic}[1]
\State{Compute the eigendecomposition $A = Q\Lambda Q\ctp$.}
\State{Compute the matrix square root $A^{1/2} = Q\Lambda^{1/2}Q\ctp$.}
\end{algorithmic}
\end{algorithm}

These two algorithms both compute the eigendecomposition of a positive
definite matrix and, by previous section, we can compute the SVD instead of
the eigendecomposition such that the accuracy of the one-sided Jacobi can
be explored.

\begin{enumerate}
\item For Algorithm~\ref{alg:sqrt-higham}, we can compute the Hermitian
polar factor of $R$ using
\begin{enumerate}
\item Scaled Newton Method, \ycite[2008, Algorithm~8.20]{high08_fm}. This
can make use of the existing code by Higham~\cite{high-mft}.
\item Newton--Schulz iteration.
\item The DV-SVD algorithm.
\end{enumerate}
\item For Algorithm~\ref{alg:sqrt-eigdecomp}, the analysis of the DV-SVD
algorithm is also required. This involves several papers by Drma{\v{c}}.
\end{enumerate}

The aim of this stage is to compare different method for computing the
matrix square root of a Hermitian positive definite matrix. This can also
extend to -- How to use mixed-precision one-sided Jacobi
algorithm~\cite{gms22} to speed up the process without loss of accuracy.


%%%%%%%%%%%%%%%%%%%%%%%%%%%%%%%%%%%%%%%%%%%%%%%%%%%%%%%%%%%%%%%%%%%%%%

\section{Summary of Reading}
\label{sec:summary-reading}

\begin{enumerate}
\item \cite{high97}\footnote{Detail notes can be found here
  \href{https://github.com/zhengbo0503/reading_notes/blob/main/high97/do.pdf}{url}}: 
  Newton's method has been used for computing the matrix  
  square root. However, this method is unstable. The paper rederive the
  stable DB iteration and derive the coupled Newton--Schulz iteration for
  matrix square root. Scaling method is briefly discussed in context of the
  new derivation. A new way for computing the square root of the positive
  definite matrix is given. 
\end{enumerate}







%%%%%%%%%%%%%%%%%%%%%%%%%%%%%%%%%%%%%%%%%%%%%%%%%%%%%%%%%%%%%%%%%%%%%%
%%%%%%%%%%%%%%%%%%%%%%%%%%% Bibliographies %%%%%%%%%%%%%%%%%%%%%%%%%%%
%%%%%%%%%%%%%%%%%%%%%%%%%%%%%%%%%%%%%%%%%%%%%%%%%%%%%%%%%%%%%%%%%%%%%%
\newpage
\bibliographystyle{/Users/clement/Dropbox/bibtex/nj-plain}
\bibliography{/Users/clement/Dropbox/bibtex/strings.bib,
  /Users/clement/Dropbox/bibtex/ref.bib} 
%% APPENDICES


\end{document}


%%% Local Variables:
%%% mode: latex
%%% TeX-master: t
%%% End:
