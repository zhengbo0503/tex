\documentclass{article}

\def\ntitle {Computing Eigensystem}
% \def\nauthor{ } % default author is Zhengbo Zhou
% \def\notes{ } % default is the submitted version
% \def\ndate{ } % default is today's date
\RequirePackage{etex}
\makeatletter
\ifx \nauthor\undefined
  \def\nauthor{Zhengbo Zhou}
\else 
\fi 
\ifx \ndate\undefined 
  \def\ndate{\today}
\else 
\fi 

\author{\nauthor \thanks{%
Department of Mathematics,
University of Manchester,
Manchester, M13 9PL, England
(\texttt{zhengbo.zhou@postgrad.manchester.ac.uk}).
}}
\date{\ndate}
\title{\ntitle}

% RedeclareMathOperator
\newcommand\RedeclareMathOperator{%
  \@ifstar{\def\rmo@s{m}\rmo@redeclare}{\def\rmo@s{o}\rmo@redeclare}%
}
% this is taken from \renew@command
\newcommand\rmo@redeclare[2]{%
  \begingroup \escapechar\m@ne\xdef\@gtempa{{\string#1}}\endgroup
  \expandafter\@ifundefined\@gtempa
     {\@latex@error{\noexpand#1undefined}\@ehc}%
     \relax
  \expandafter\rmo@declmathop\rmo@s{#1}{#2}}
% This is just \@declmathop without \@ifdefinable
\newcommand\rmo@declmathop[3]{%
  \DeclareRobustCommand{#2}{\qopname\newmcodes@#1{#3}}%
}
\@onlypreamble\RedeclareMathOperator

\usepackage{algorithm}
\usepackage{algpseudocode}
\usepackage{comment}
\usepackage{bookmark}
\usepackage{microtype}
\usepackage{booktabs}
\usepackage{lastpage}
\usepackage{fancyhdr}
\usepackage{amsthm}
\usepackage{mathtools}
\usepackage{enumerate}
\usepackage{mathrsfs}
\usepackage{amsfonts}
\usepackage{amssymb}
\usepackage{tcolorbox}
\usepackage{bm}
\usepackage{cancel}
\usepackage{bbm}
\usepackage{accsupp}
\usepackage{enumitem}
\usepackage{hyperref}
\usepackage[fontsize=12pt]{fontsize}
\usepackage{geometry}
\usepackage{microtype}
\usepackage{algorithm,algorithmicx,algpseudocode}

\hbadness=99999

\geometry{
  a4paper,
  textwidth=165truemm,
  textheight=240truemm,
  top=28.5truemm,
}

\let\LaTeXStandardTableOfContents\tableofcontents
\renewcommand{\tableofcontents}{%
\begingroup%
\renewcommand{\bfseries}{\sc}%
\LaTeXStandardTableOfContents%
\endgroup%
}%
\setcounter{tocdepth}{3}
\makeatletter
\renewcommand\tableofcontents{\@starttoc{toc}}
\makeatother

\hypersetup{
  hypertexnames=false, 
  colorlinks=true,
  linkcolor=blue,
  pdfauthor={Zhengbo Zhou},
  pdftitle={\ntitle},
  pdfcreator={Zhengbo Zhou via MikTeX}
}

\usepackage[hyperpageref]{backref}
\renewcommand*{\backref}[1]{}
\renewcommand*{\backrefalt}[4]{
    \ifcase #1 %
    No citations.%
    \or
    (Cited on p.~#2.)
    \else
    (Cited on pp.~#2.)
    \fi
}



\pagestyle{fancyplain}
\fancyhead[R]{{\textit{\footnotesize\nouppercase Short Course : Function of Matrices}}}
\fancyhead[L]{\footnotesize\nouppercase\leftmark}


\newtcolorbox{mybox}[1]{colback=white!90!black,colframe=white!40!black,fonttitle=\scshape\centering,title=#1}

%%% MATLAB Code From Dr. Chris Johnson 
\usepackage{color}  
\usepackage{xcolor}
\usepackage{listings}
\definecolor{codegreen}{rgb}{0,0.6,0}
\definecolor{codegray}{rgb}{0.5,0.5,0.5}
\definecolor{codepurple}{rgb}{0.58,0,0.82}
\definecolor{mygreen}{RGB}{28,172,0}
\definecolor{mylilas}{RGB}{170,55,241}
\definecolor{backcolour}{rgb}{0.95,0.95,1.92}
\lstdefinestyle{mystyle}{
	language=matlab,
    commentstyle=\color{codegreen},
    keywordstyle=\color{blue},
    numberstyle=\footnotesize\color{codegray},
    stringstyle=\color{codepurple},
    basicstyle=\linespread{1}\ttfamily\small,
    breakatwhitespace=false,
    breaklines=true,
    captionpos=b,
    keepspaces=true,
    numbers=left,
    numbersep=10pt,
    showspaces=false,
    showstringspaces=false,
    showtabs=false,
    tabsize=4,
    aboveskip=\medskipamount,
    % frame=single,
}
\lstset{style=mystyle}
\def\inline{\lstinline[basicstyle=\upshape\ttfamily]}

%% Theorems 
\newtheorem{theorem}{Theorem}[section]
\newtheorem{proposition}[theorem]{Proposition}
\newtheorem{corollary}[theorem]{Corollary}
\newtheorem{lemma}[theorem]{Lemma}
\theoremstyle{definition}
\newtheorem{definition}[theorem]{Definition}
\newtheorem{example}[theorem]{Example}
\newtheorem{remark}[theorem]{Remark}
\newtheorem*{question}{Question}
\newtheorem*{recall}{Recall}
\newtheorem*{assumption}{Assumption}
\newtheorem*{note}{Note}
\numberwithin{equation}{section}

\let\stdsection\section
\renewcommand\section{\newpage\stdsection}

%%% Paired labels
\DeclarePairedDelimiter\ceil{\lceil}{\rceil}
\DeclarePairedDelimiter\floor{\lfloor}{\rfloor} 
\DeclarePairedDelimiter\abs{\lvert}{\rvert} % |a|
\DeclarePairedDelimiter\inner{\langle}{\rangle} % <a>
\makeatletter
\let\oldabs\abs
\def\abs{\@ifstar{\oldabs}{\oldabs*}}

%%%------------------------------------------------------------------%%%

%%% vector bold
\renewcommand{\vec}[1]{\bm{#1}}

%%%------------------------------------------------------------------%%%

%%% Real and Imaginary
\RedeclareMathOperator{\Im}{\mathrm{Im}}
\RedeclareMathOperator{\Re}{\mathrm{Re}}

%%% Integrate from ... to ...       
\newcommand{\intii}{\int_{-\infty}^{\infty}}

%%% Greek Letters
% NEVER define \l for \lambda due to Polish names in BibTeX
\renewcommand{\L}{\Lambda}
\newcommand{\vL}{\varLambda}
\newcommand{\g}{\gamma}
\newcommand{\G}{\Gamma}
\newcommand{\vG}{\varGamma}
\renewcommand{\o}{\omega}
\renewcommand{\O}{\Omega}
\newcommand{\vO}{\varOmega}
\newcommand{\s}{\sigma}
\renewcommand{\S}{\Sigma}
\newcommand{\vS}{\varSigma}
\newcommand{\eps}{\varepsilon}
\newcommand{\lap}{\varDelta}

%%% Matrix Related
\newcommand{\n}{^{n}}
\newcommand{\m}{^{m}}
\newcommand{\Rn}{\R^n}
\newcommand{\mn}{^{m\times n}}
\newcommand{\nn}{^{n\times n}}
\newcommand{\tp}{^{T}} 
\newcommand{\ctp}{^{*}}
\newcommand{\inv}{^{-1}}
\DeclareMathOperator{\diag}{diag}
\DeclareMathOperator{\rank}{rank}
\DeclareMathOperator{\tr}{trace}
\DeclareMathOperator{\range}{Range}

%%%------------------------------------------------------------------%%%

%% Norms
\newcommand{\iter}[1]{^{(#1)}} % iteration
\newcommand{\gnorm}[1]{\left\|{#1}\right\|} % general norm
\newcommand{\norm}[1]{\gnorm{#1}}
\newcommand{\tnorm}[1]{\gnorm{#1}_2} % 2-norm
\newcommand{\inorm}[1]{\gnorm{#1}_\infty} % infinity norm

%%% Over the expressions
\newcommand{\wh}{\widehat}
\newcommand{\wt}{\widetilde}
\newcommand{\wb}{\overline}

%%% Calculus
\DeclareMathOperator{\grad}{\nabla}
\renewcommand{\div}{\nabla\cdot}
\DeclareMathOperator{\curl}{\nabla\times}
\newcommand{\dd}{\mathrm{d}}
\newcommand{\pp}{\partial}
\def\eu{\mathrm{e}} % euler's constant
\def\im{\mathrm{i}} % imaginary unit

%%% Citation
\def\ycite[#1#2#3#4#5]#6{\cite[$\mit{#1#2#3#4}$#5]{#6}}

%%%------------------------------------------------------------------%%%

%%% MATHBB
\newcommand{\mb}[1]{\mathbb{#1}}
\newcommand{\N}{\mb{N}}
\newcommand{\Z}{\mb{Z}} 
\newcommand{\Q}{\mb{Q}}
\newcommand{\R}{\mb{R}} 
\newcommand{\C}{\mb{C}}
\newcommand{\F}{\mb{F}}
\renewcommand{\P}{\mb{P}} % Probability
\newcommand{\E}{\mb{E}} % Expectation
\newcommand{\V}{\mb{V}} % Variance

%%% MATHCAL and MATHSCR
\newcommand{\mc}[1]{\mathcal{#1}} % For spaces 
\newcommand{\ms}[1]{\mathscr{#1}} % For sigma-algebra
\newcommand{\mf}[1]{\mathfrak{#1}}

%%%------------------------------------------------------------------%%%

%%% Stochastic Calculus
\newcommand{\ps}{$(\Omega,\ms F,\P)$}
\newcommand{\ito}{It\^o}
\DeclareMathVersion{bold}
\newcommand{\indi}{\mathbbm{1}} % indicator function
\providecommand*{\napprox}{%
  \BeginAccSupp{method=hex,unicode,ActualText=2249}%
  \not\approx
  \EndAccSupp{}%
}
\mathchardef\gang="2D
\DeclareMathOperator*{\esssup}{esssup}

\DeclareMathOperator{\law}{\mathfrak{Law}}
\newcommand{\loc}{\textup{loc}}
\newcommand{\locmart}{\mc M_{\loc}^C}
\newcommand{\locm}{\mc M_{\loc,T}^{C,0}}
\newcommand{\locmt}{\wt{\mc M}_{\loc,T}^{C,0}}
\newcommand{\nicee}[1]{\mathcal{E}_{#1}^H(B)}


\DeclareMathOperator{\sign}{diag}

\makeatother

\begin{document}
\maketitle
% \tableofcontents
\thispagestyle{firstpage} 

%%%%%%%%%%%%%%%%%%%%%%%%%%%% MAIN ARTICLE %%%%%%%%%%%%%%%%%%%%%%%%%%%%
\section{Divide and Conquer}
\label{sec:divide-conquer}
\subsection{Introduction}
This is currently the fastest method to find all the eigenvalues and
eigenvectors of symmetric tridiagonal matrices larger than $n = 25$.
In the worst cases, divide-and-conquer requires $O(n^3)$ flops, but in
practice, the constant is quite small. 
Over a large of set of random test cases, it appears to take only
$O(n^{2.3})$ flops on average, and as low as $O(n^2)$ for some eigenvalue
distribution.

In theory, divide-and-conquer could be implemented to run on $O(n\log^pn)$
flops, where $p$ is a small integer. This super-fast implementation uses
the fast multi-pole method (FMM), originally invented for the completely
different problem of computing the mutual forces on $n$ electrically
charged particles.
But the complexity of this super-fast implementation means that QR
iteration is currently the algorithm of choice for finding all eigenvalues,
and divide-and-conquer without the FMM is the method of choice for finding
all eigenvalues and all eigenvectors.

\subsection{Overview}
It is quite subtle to implement in a numerically stable way. Indeed,
although this method was first introduced in 1981, the ``right''
implementation was not discovered until 1992.  This routine is available as
LAPACK routines \texttt{ssyevd} for dense matrices and \texttt{sstevd} for
tridiagonal matrices.  This routine uses divide-and-conquer for matrices of
dimension larger than 25 and automatically switches to QR iteration for
smaller matrices.

\begin{equation}
  \begin{aligned}
    T & =
        \left[
        \begin{array}{cccc|cccc}
          a_1 & b_1 & & & & & & \\
          b_1 & \ddots & \ddots & & & & & \\
              & \ddots & a_{m-1} & b_{m-1} & & & & \\
              & & b_{m-1} & a_m & b_m & & & \\
          \hline & & & b_m & a_{m+1} & b_{m+1} & & \\
              & & & & b_{m+1} & \ddots & & \\
              & & & & & & \ddots & b_{n-1} \\
              & & & & & & b_{n-1} & a_n
        \end{array}\right]\\
      & =\left[
        \begin{array}{cccc|cccc}
          a_1 & b_1 & & & & & & \\
          b_1 & \ddots & \ddots & & & & \\
              & \ddots & a_{m-1} & b_{m-1} & & & & \\
              & & b_{m-1} & a_m-b_m & & & \\
          \hline & & & & a_{m+1}-b_m & b_{m+1} & & \\
              & & & & b_{m+1} & \ddots & & \\
              & & & & & \ddots & b_{n-1} \\
              & & & & & b_{n-1} & a_n
        \end{array}\right]
        + \left[
        \begin{array}{cc|cc}
          & 	&  &  \\
          & b_m & b_m &  \\\hline
          & b_m & b_m &  \\
          &  & &
        \end{array}\right] \\
      & = \left[\begin{array}{c|c}
                  T_1 & 0 \\
                  \hline 0 & T_2
                \end{array}
        \right]+b_m \cdot\left[
        \begin{array}{c}
          0 \\
          \vdots \\
          0 \\
          1 \\
          1 \\
          0 \\
          \vdots \\
          0
        \end{array}\right][0, \ldots, 0,1,1,0, \ldots, 0] \equiv\left[
        \begin{array}{c|c}
          T_1 & 0 \\
          \hline 0 & T_2
        \end{array}\right]+b_m v v^T .
  \end{aligned}
\end{equation}
Suppose that we have the eigendecomposition of $T_1$ and $T_2$:
$T_i = Q_i\Lambda_iQ_i\tp$. These will be computed recursively by this same
algorithm. We related the eigenvalues of $T$ to those of $T_1$ and $T_2$ as
follows:
\begin{equation}
  \notag
  \begin{aligned}
    T & =
        \begin{bmatrix}
          T_1 & 0 \\ 0 & T_2
        \end{bmatrix}
        + b_mvv\tp \\
      & =
        \begin{bmatrix}
          Q_1\Lambda_1Q_1\tp & 0 \\ 0 & Q_2\Lambda_2Q_2\tp
        \end{bmatrix}
        + b_mvv\tp\\
      & =
        \begin{bmatrix}
          Q_1 & 0 \\ 0 & Q_2
        \end{bmatrix}
        \left(
        \begin{bmatrix}
          \Lambda_1 & \\ & \Lambda_2
        \end{bmatrix} + b_muu\tp
        \right)
        \begin{bmatrix}
          Q_1\tp & 0 \\ 0 & Q_2\tp
        \end{bmatrix}
  \end{aligned}
\end{equation}
where
\begin{equation*}
  u =
  \begin{bmatrix}
    Q_1\tp & 0 \\ 0 & Q_2
  \end{bmatrix}, \qquad
  v =
  \begin{bmatrix}
    \text{last column of $Q_1\tp$} \\ \text{first column of $Q_2\tp$}
  \end{bmatrix}
\end{equation*}
since $v = [0,\dots,0,1,1,0,\dots,0]\tp$. Therefore, the eigenvalues of $T$
are the same as those of the similar matrix $D + \rho uu\tp$ where $D =
\begin{bmatrix}
  \Lambda_1 & 0 \\ 0 & \Lambda_2
\end{bmatrix}
$ is diagonal, $\rho = b_m$ is a scalar, and $u$ is a vector. Henceforth,
we will assume without loss of generality that the diagonal
$d_1,\cdots,d_n$ of $D$ is sorted: $d_n\leq\cdots\leq d_n$.

To find the eigenvalues of $D + \rho uu\tp$, assume first that
$D - \lambda I$ is nonsingular, and compute the characteristic polynomial
as follows:
\begin{equation}
  \label{eq:3}
  \det(D + \rho uu^T - \lambda I) 
  = \det((D - \lambda I)(I + \rho(D - \lambda I )^{-1}uu^T)).
\end{equation}

Since $D - \lambda I$ is nonsingular, $\det(I + \rho(D -
\lambda I)^{-1}uu^T = 0$ whenever $\lambda$ is an eigenvalue. Notice that
$I+\rho(D - \lambda I)^{-1}uu^T$ is the identity plus a rank-1 matrix; the
determinant of such a matrix is easy to compute:

\begin{lemma}
  If $x$ and $y$ are vectors, $\det(I + xy^T) = 1 + y^Tx$.
\end{lemma}
Therefore
\begin{equation}\notag
  \begin{aligned}
    \det(I + \rho(D - \lambda I)^{-1}uu^T)  & = 1 + \rho u^T(D - \lambda
                                              I)^{-1}u \\
      & = 1 + \rho\sum_{i=1}^n \frac{u_i^2}{d_i-\lambda} \equiv f(\lambda). 
  \end{aligned}
\end{equation}
and the eigenvalues of $T$ are the roots of the so-called secular equation
$f(\lambda) = 0$. If all $d_i$ are distinct and all $u_i\neq 0$ (the
generic case), the function $f(\lambda)$ has the graph shown in.



%%%%%%%%%%%%%%%%%%%%%%%%%%% Bibliographies %%%%%%%%%%%%%%%%%%%%%%%%%%%
\newpage
\bibliographystyle{nj-plain}
\bibliography{~/bibtex/references.bib}


\end{document}

%%% Local Variables:
%%% mode: latex
%%% TeX-master: t
%%% End:
