%%%%%%%%%%%%%%%%%%%%%%%%%%%%%%%%%%%%%%%%%%%%%%%%%%%%%%%%%%%%%%%%%%%%%
%%%%%%%%%%%%%%%%%%%%%%%% Systematic Settings %%%%%%%%%%%%%%%%%%%%%%%%
%%%%%%%%%%%%%%%%%%%%%%%%%%%%%%%%%%%%%%%%%%%%%%%%%%%%%%%%%%%%%%%%%%%%%

% DEFINE ntitle/nauthor/notes/ndate
\RequirePackage{etex}
\makeatletter
\ifx \nauthor\undefined
  \def\nauthor{Zhengbo Zhou}
\else 
\fi 
\ifx \ndate\undefined 
  \def\ndate{\today}
\else 
\fi 
\ifx \notes\undefined 
\else
    \makeatletter
    \renewcommand\maketitle
        {\noindent
            {\large\scshape\ntitle}
            \medskip\par\noindent
            {\slshape\nauthor}
            \hfill
            {\normalsize\itshape\ndate}
            \bigskip\par\noindent
        }
    \makeatother
\fi

\author{\nauthor \thanks{
Department of Mathematics,
University of Manchester,
Manchester, M13 9PL, England
(\texttt{zhengbo.zhou@postgrad.manchester.ac.uk}).
}}
\date{\ndate}
\title{\ntitle}

%% Required Packages
\usepackage{algorithm}
\usepackage{algpseudocode}
\usepackage{comment}
\usepackage{bookmark}
\usepackage{microtype}
\usepackage[dvipsnames]{xcolor}
\usepackage{booktabs}
\usepackage{lastpage}
\usepackage{fancyhdr}
\usepackage{amsthm}
\usepackage{mathtools}
\usepackage{enumerate}
\usepackage{mathrsfs}
\usepackage{amsfonts}
\usepackage{amssymb}
\usepackage{tcolorbox}
\usepackage{bm}
\usepackage{cancel}
\usepackage{bbm}
\usepackage{accsupp}
\usepackage{enumitem}
\usepackage{hyperref}
\usepackage[fontsize=12pt]{fontsize}
\usepackage{geometry}
\usepackage{microtype}
\usepackage{anyfontsize}
\usepackage[hyperpageref]{backref}

%% Avoid badscalings
\renewcommand\hmmax{0}
\renewcommand\bmmax{0}
\hbadness=99999

%% page size
\geometry{
  a4paper,
  textwidth=159truemm,
  textheight=240truemm,
  top=24.5truemm,
}

%% table of content settings
\setcounter{tocdepth}{2}

%% hyperref and backref
\hypersetup{
    hypertexnames=false, 
    colorlinks,
    linkcolor={red},
    citecolor={red},
    urlcolor={blue},
    pdfauthor={Zhengbo Zhou},
    pdftitle={\ntitle},
    pdfcreator={Zhengbo Zhou via MikTeX}
}
\renewcommand*{\backref}[1]{}
\renewcommand*{\backrefalt}[4]{
    \ifcase #1 
    No citations.
    \or
    (Cited on p.~#2.)
    \else
    (Cited on pp.~#2.)
    \fi
}

%% header and footer
\fancypagestyle{plain}{
    \fancyhf{}
    \renewcommand{\headrulewidth}{0pt}
    \lhead{\textit{\nouppercase{\leftmark}}}
    \rhead{\ntitle}
    \cfoot{\thepage\ of \pageref*{LastPage}}
}
\fancypagestyle{firstpage}[plain]{
    \renewcommand{\headrulewidth}{0pt}
    \fancyhead[L,C,R]{}
    \fancyfoot[L,C,R]{}
}
\pagestyle{plain}

%% colorboxes
\newtcolorbox{mybox}[1]{ % grey background box
    colback=white!90!black,
    colframe=white!40!black,
    fonttitle=\scshape\centering,
    title=#1
}
\newtcolorbox{solution}[1]{ % pure white box with border
    colback=white,
    colframe=black,
    boxrule=0.75pt,
    breakable,
    arc=0pt,
    outer arc=0pt,
    fonttitle=\scshape\bfseries
} 

%% redefine \emph and \bf such that they are colored and can be easily
%% transformed back to normal \emph and \bf
\renewcommand{\emph}[1]{\textit{\color{violet} #1}}
\renewcommand{\bf}[1]{\textsf{\bfseries \color{violet} #1}}

%% Theorems 
\newtheorem{theorem}{Theorem}[section] % Italic font 
\newtheorem{proposition}[theorem]{Proposition}
\newtheorem{corollary}[theorem]{Corollary}
\newtheorem{lemma}[theorem]{Lemma}

\theoremstyle{definition}
\newtheorem{definition}[theorem]{Definition} % upright font
\newtheorem{example}[theorem]{Example}
\newtheorem{remark}[theorem]{Remark}
\newtheorem*{question}{Question}
\newtheorem*{recall}{Recall}
\newtheorem*{assumption}{Assumption}
\newtheorem*{note}{Note}

\numberwithin{equation}{section} % equation numbers

%% Font 
\usepackage{anyfontsize}
\usepackage{newtxtext}     
\usepackage[varvw]{newtxmath} 
\def\ttdefault{cmtt}
\usepackage[ bb=pazo ]{mathalpha}

%%%%%%%%%%%%%%%%%%%%%%%%%%%%%%%%%%%%%%%%%%%%%%%%%%%%%%%%%%%%%%%%%%%%%
%%%%%%%%%%%%%%%%%%%%%%%% USER DEFINED MACROS %%%%%%%%%%%%%%%%%%%%%%%%
%%%%%%%%%%%%%%%%%%%%%%%%%%%%%%%%%%%%%%%%%%%%%%%%%%%%%%%%%%%%%%%%%%%%%

%% MATHBB
\newcommand{\mb}[1]{\mathbb{#1}}
\newcommand{\N}{\mb{N}}
\newcommand{\Z}{\mb{Z}} 
\newcommand{\Q}{\mb{Q}}
\newcommand{\R}{\mb{R}} 
\newcommand{\C}{\mb{C}}
\newcommand{\F}{\mb{F}}
\renewcommand{\P}{\mb{P}} %% Probability
\newcommand{\E}{\mb{E}} %% Expectation
\newcommand{\V}{\mb{V}} %% Variance

%%%%%% MATHCAL and MATHSCR
\newcommand{\mc}[1]{\mathcal{#1}} %% For spaces 
\newcommand{\ms}[1]{\mathscr{#1}} %% For sigma-algebra

%% Paired labels
\DeclarePairedDelimiter\ceil{\lceil}{\rceil} % ceiling 
\DeclarePairedDelimiter\floor{\lfloor}{\rfloor}  % floors
\DeclarePairedDelimiter\inner{\langle}{\rangle} %% <a>
\DeclarePairedDelimiter\abs{\lvert}{\rvert} %% |a|
\makeatletter
\let\oldabs\abs
\def\abs{\@ifstar{\oldabs}{\oldabs*}}

%% vector bold
\renewcommand{\vec}[1]{\bm{#1}}

%% Real and Imaginary
\renewcommand{\Re}[1]{\mathrm{Re}\left\{{#1}\right\}}
\renewcommand{\Im}[1]{\mathrm{Im}\left\{{#1}\right\}}

%% Integrate from ... to ...
\newcommand{\intii}{\int_{-\infty}^{\infty}}

%% Greek Letters
% NEVER define \l for \lambda due to Polish names in BibTeX
\newcommand{\eps}{\varepsilon}

%% Matrix Analysis
\newcommand{\Rn}{\R^n}
\newcommand{\Cn}{\C^n}
\newcommand{\n}{^{n}}
\newcommand{\m}{^{m}}
\newcommand{\mn}{^{m\times n}}
\newcommand{\nn}{^{n\times n}}
\newcommand{\tp}{^{T}} % transpose 
\newcommand{\ctp}{^{*}} % conjugate transpose
\newcommand{\inv}{^{-1}}
\DeclareMathOperator{\diag}{diag}
\DeclareMathOperator{\rank}{rank}
\DeclareMathOperator{\tr}{trace}
\DeclareMathOperator{\range}{Range}
\newcommand{\iter}[1]{^{(#1)}} %% iteration \iter{}

%% Norms
\newcommand{\norm}[1]{\left\|{#1}\right\|}
\newcommand{\fnorm}[1]{\norm{#1}_F}
\newcommand{\tnorm}[1]{\norm{#1}_2} % 2-norm
\newcommand{\inorm}[1]{\norm{#1}_\infty} % infinity norm

%% Over the expressions
\newcommand{\wh}{\widehat}
\newcommand{\wt}{\widetilde}
\newcommand{\wb}{\overline}

%% Calculus
\newcommand{\lap}{\varDelta}
\DeclareMathOperator{\grad}{\nabla}
\renewcommand{\div}{\nabla\cdot}
\DeclareMathOperator{\curl}{\nabla\times}
\newcommand{\dd}{\mathrm{d}}
\newcommand{\pp}{\partial}
\def\eu{\mathrm{e}} % euler's constant
\def\im{\mathrm{i}} % imaginary unit
\DeclareMathVersion{bold}
\newcommand{\indi}{\mathbbm{1}} %% indicator function

%% Citation % nick
\def\ycite[#1#2#3#4#5]#6{\cite[$\mit{#1#2#3#4}$#5]{#6}}

% newtxfont 
\usepackage{anyfontsize}
\usepackage{newtxtext}     
\usepackage[varvw]{newtxmath} 
\def\ttdefault{cmtt}

\makeatother