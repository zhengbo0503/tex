\documentclass{article}

\def\ntitle {Notes on papers }
% \def\nauthor{ } % default author is Zhengbo Zhou
% \def\notes{ } % default is the submitted version
% \def\ndate{ } % default is today's date
\RequirePackage{etex}
\makeatletter
\ifx \nauthor\undefined
  \def\nauthor{Zhengbo Zhou}
\else 
\fi 
\ifx \ndate\undefined 
  \def\ndate{\today}
\else 
\fi 

\author{\nauthor \thanks{%
Department of Mathematics,
University of Manchester,
Manchester, M13 9PL, England
(\texttt{zhengbo.zhou@postgrad.manchester.ac.uk}).
}}
\date{\ndate}
\title{\ntitle}

% RedeclareMathOperator
\newcommand\RedeclareMathOperator{%
  \@ifstar{\def\rmo@s{m}\rmo@redeclare}{\def\rmo@s{o}\rmo@redeclare}%
}
% this is taken from \renew@command
\newcommand\rmo@redeclare[2]{%
  \begingroup \escapechar\m@ne\xdef\@gtempa{{\string#1}}\endgroup
  \expandafter\@ifundefined\@gtempa
     {\@latex@error{\noexpand#1undefined}\@ehc}%
     \relax
  \expandafter\rmo@declmathop\rmo@s{#1}{#2}}
% This is just \@declmathop without \@ifdefinable
\newcommand\rmo@declmathop[3]{%
  \DeclareRobustCommand{#2}{\qopname\newmcodes@#1{#3}}%
}
\@onlypreamble\RedeclareMathOperator

\usepackage{algorithm}
\usepackage{algpseudocode}
\usepackage{comment}
\usepackage{bookmark}
\usepackage{microtype}
\usepackage{booktabs}
\usepackage{lastpage}
\usepackage{fancyhdr}
\usepackage{amsthm}
\usepackage{mathtools}
\usepackage{enumerate}
\usepackage{mathrsfs}
\usepackage{amsfonts}
\usepackage{amssymb}
\usepackage{tcolorbox}
\usepackage{bm}
\usepackage{cancel}
\usepackage{bbm}
\usepackage{accsupp}
\usepackage{enumitem}
\usepackage{hyperref}
\usepackage[fontsize=12pt]{fontsize}
\usepackage{geometry}
\usepackage{microtype}
\usepackage{algorithm,algorithmicx,algpseudocode}

\hbadness=99999

\geometry{
  a4paper,
  textwidth=165truemm,
  textheight=240truemm,
  top=28.5truemm,
}

\let\LaTeXStandardTableOfContents\tableofcontents
\renewcommand{\tableofcontents}{%
\begingroup%
\renewcommand{\bfseries}{\sc}%
\LaTeXStandardTableOfContents%
\endgroup%
}%
\setcounter{tocdepth}{3}
\makeatletter
\renewcommand\tableofcontents{\@starttoc{toc}}
\makeatother

\hypersetup{
  hypertexnames=false, 
  colorlinks=true,
  linkcolor=blue,
  pdfauthor={Zhengbo Zhou},
  pdftitle={\ntitle},
  pdfcreator={Zhengbo Zhou via MikTeX}
}

\usepackage[hyperpageref]{backref}
\renewcommand*{\backref}[1]{}
\renewcommand*{\backrefalt}[4]{
    \ifcase #1 %
    No citations.%
    \or
    (Cited on p.~#2.)
    \else
    (Cited on pp.~#2.)
    \fi
}



\pagestyle{fancyplain}
\fancyhead[R]{{\textit{\footnotesize\nouppercase Short Course : Function of Matrices}}}
\fancyhead[L]{\footnotesize\nouppercase\leftmark}


\newtcolorbox{mybox}[1]{colback=white!90!black,colframe=white!40!black,fonttitle=\scshape\centering,title=#1}

%%% MATLAB Code From Dr. Chris Johnson 
\usepackage{color}  
\usepackage{xcolor}
\usepackage{listings}
\definecolor{codegreen}{rgb}{0,0.6,0}
\definecolor{codegray}{rgb}{0.5,0.5,0.5}
\definecolor{codepurple}{rgb}{0.58,0,0.82}
\definecolor{mygreen}{RGB}{28,172,0}
\definecolor{mylilas}{RGB}{170,55,241}
\definecolor{backcolour}{rgb}{0.95,0.95,1.92}
\lstdefinestyle{mystyle}{
	language=matlab,
    commentstyle=\color{codegreen},
    keywordstyle=\color{blue},
    numberstyle=\footnotesize\color{codegray},
    stringstyle=\color{codepurple},
    basicstyle=\linespread{1}\ttfamily\small,
    breakatwhitespace=false,
    breaklines=true,
    captionpos=b,
    keepspaces=true,
    numbers=left,
    numbersep=10pt,
    showspaces=false,
    showstringspaces=false,
    showtabs=false,
    tabsize=4,
    aboveskip=\medskipamount,
    % frame=single,
}
\lstset{style=mystyle}
\def\inline{\lstinline[basicstyle=\upshape\ttfamily]}

%% Theorems 
\newtheorem{theorem}{Theorem}[section]
\newtheorem{proposition}[theorem]{Proposition}
\newtheorem{corollary}[theorem]{Corollary}
\newtheorem{lemma}[theorem]{Lemma}
\theoremstyle{definition}
\newtheorem{definition}[theorem]{Definition}
\newtheorem{example}[theorem]{Example}
\newtheorem{remark}[theorem]{Remark}
\newtheorem*{question}{Question}
\newtheorem*{recall}{Recall}
\newtheorem*{assumption}{Assumption}
\newtheorem*{note}{Note}
\numberwithin{equation}{section}

\let\stdsection\section
\renewcommand\section{\newpage\stdsection}

%%% Paired labels
\DeclarePairedDelimiter\ceil{\lceil}{\rceil}
\DeclarePairedDelimiter\floor{\lfloor}{\rfloor} 
\DeclarePairedDelimiter\abs{\lvert}{\rvert} % |a|
\DeclarePairedDelimiter\inner{\langle}{\rangle} % <a>
\makeatletter
\let\oldabs\abs
\def\abs{\@ifstar{\oldabs}{\oldabs*}}

%%%------------------------------------------------------------------%%%

%%% vector bold
\renewcommand{\vec}[1]{\bm{#1}}

%%%------------------------------------------------------------------%%%

%%% Real and Imaginary
\RedeclareMathOperator{\Im}{\mathrm{Im}}
\RedeclareMathOperator{\Re}{\mathrm{Re}}

%%% Integrate from ... to ...       
\newcommand{\intii}{\int_{-\infty}^{\infty}}

%%% Greek Letters
% NEVER define \l for \lambda due to Polish names in BibTeX
\renewcommand{\L}{\Lambda}
\newcommand{\vL}{\varLambda}
\newcommand{\g}{\gamma}
\newcommand{\G}{\Gamma}
\newcommand{\vG}{\varGamma}
\renewcommand{\o}{\omega}
\renewcommand{\O}{\Omega}
\newcommand{\vO}{\varOmega}
\newcommand{\s}{\sigma}
\renewcommand{\S}{\Sigma}
\newcommand{\vS}{\varSigma}
\newcommand{\eps}{\varepsilon}
\newcommand{\lap}{\varDelta}

%%% Matrix Related
\newcommand{\n}{^{n}}
\newcommand{\m}{^{m}}
\newcommand{\Rn}{\R^n}
\newcommand{\mn}{^{m\times n}}
\newcommand{\nn}{^{n\times n}}
\newcommand{\tp}{^{T}} 
\newcommand{\ctp}{^{*}}
\newcommand{\inv}{^{-1}}
\DeclareMathOperator{\diag}{diag}
\DeclareMathOperator{\rank}{rank}
\DeclareMathOperator{\tr}{trace}
\DeclareMathOperator{\range}{Range}

%%%------------------------------------------------------------------%%%

%% Norms
\newcommand{\iter}[1]{^{(#1)}} % iteration
\newcommand{\gnorm}[1]{\left\|{#1}\right\|} % general norm
\newcommand{\norm}[1]{\gnorm{#1}}
\newcommand{\tnorm}[1]{\gnorm{#1}_2} % 2-norm
\newcommand{\inorm}[1]{\gnorm{#1}_\infty} % infinity norm

%%% Over the expressions
\newcommand{\wh}{\widehat}
\newcommand{\wt}{\widetilde}
\newcommand{\wb}{\overline}

%%% Calculus
\DeclareMathOperator{\grad}{\nabla}
\renewcommand{\div}{\nabla\cdot}
\DeclareMathOperator{\curl}{\nabla\times}
\newcommand{\dd}{\mathrm{d}}
\newcommand{\pp}{\partial}
\def\eu{\mathrm{e}} % euler's constant
\def\im{\mathrm{i}} % imaginary unit

%%% Citation
\def\ycite[#1#2#3#4#5]#6{\cite[$\mit{#1#2#3#4}$#5]{#6}}

%%%------------------------------------------------------------------%%%

%%% MATHBB
\newcommand{\mb}[1]{\mathbb{#1}}
\newcommand{\N}{\mb{N}}
\newcommand{\Z}{\mb{Z}} 
\newcommand{\Q}{\mb{Q}}
\newcommand{\R}{\mb{R}} 
\newcommand{\C}{\mb{C}}
\newcommand{\F}{\mb{F}}
\renewcommand{\P}{\mb{P}} % Probability
\newcommand{\E}{\mb{E}} % Expectation
\newcommand{\V}{\mb{V}} % Variance

%%% MATHCAL and MATHSCR
\newcommand{\mc}[1]{\mathcal{#1}} % For spaces 
\newcommand{\ms}[1]{\mathscr{#1}} % For sigma-algebra
\newcommand{\mf}[1]{\mathfrak{#1}}

%%%------------------------------------------------------------------%%%

%%% Stochastic Calculus
\newcommand{\ps}{$(\Omega,\ms F,\P)$}
\newcommand{\ito}{It\^o}
\DeclareMathVersion{bold}
\newcommand{\indi}{\mathbbm{1}} % indicator function
\providecommand*{\napprox}{%
  \BeginAccSupp{method=hex,unicode,ActualText=2249}%
  \not\approx
  \EndAccSupp{}%
}
\mathchardef\gang="2D
\DeclareMathOperator*{\esssup}{esssup}

\DeclareMathOperator{\law}{\mathfrak{Law}}
\newcommand{\loc}{\textup{loc}}
\newcommand{\locmart}{\mc M_{\loc}^C}
\newcommand{\locm}{\mc M_{\loc,T}^{C,0}}
\newcommand{\locmt}{\wt{\mc M}_{\loc,T}^{C,0}}
\newcommand{\nicee}[1]{\mathcal{E}_{#1}^H(B)}


\DeclareMathOperator{\sign}{diag}

\makeatother

\newcommand{\code}[1]{\texttt{\color{green!30!black} #1}}
\newcommand{\Egap}[1]{\min_{\lambda_i({#1}) \neq \lambda_j({#1})} \abs{\lambda_i({#1}) - \lambda_j({#1})}}

\begin{document}
\maketitle
% \tableofcontents
\thispagestyle{firstpage} 

%%%%%%%%%%%%%%%%%%%%%%%%%%%% MAIN ARTICLE %%%%%%%%%%%%%%%%%%%%%%%%%%%%
\section{Paper~{\ycite[2022]{2022ZhangBai_mixedprecisionJacobi}}}
The authors try to implement a Jacobi algorithm which utilizes the
benefit of fastness of low precision. It first computes the
approximate eigendecomposition in low precision, then orthogonalize
it using MGS approach and apply to original matrix as a
preconditioner, and finally compute the eigensystem of
preconditioned matrix. 

The paper also provides the bound on distance
of low precision matrix of eigenvector and its orthogonal QR factor, a
bound 
on $\mathrm{off}(\text{Preconditioned Matrix})$, and a sufficient
condition for the Jacobi algorithm to have quadratic convergence
that based on both $\operatorname{Egap}(A)$ and
$\operatorname{Egap}(Q\tp AQ)$, where 
\begin{equation}\notag
    \operatorname{Egap}(A) = \Egap{A}
\end{equation}
% 2. What were the key elements of the approach
The key element is the preconditioning method. Notice that the
quadratic convergence is readily presented by~\cite{1966Kempen_quadraticconvergencespecial}, and
by preconditioning, the preconditioned matrix is automatically
satisfies the condition presented in
\cite{1966Kempen_quadraticconvergencespecial}.

One improvement could be construct a simpler proof of the theorems,
especially for the bound of $\norm{Z-Q}$ where $Z$ is constructed by low
precision eigensolver and $Q$ is its orthogonal QR factor.

\subsection{Questions}

\noindent$\star$ I am confused about the following inequality arises
from \ycite[2022,p.~9, Eq.~4.9]{2022ZhangBai_mixedprecisionJacobi}:
\begin{equation}\notag
    \abs{r_{ij}} \leq \norm{R\inv - R\tp}. \qquad \text{\small\sffamily\color{purple} How is this inequality arises?}
\end{equation}

\noindent$\star$ Moreover, do we have the following inequality: suppose
$\prod_{i=1}^n A_i$ is well-defined, then for any $j\in\{1,\dots,n\}$,
\begin{equation}\notag
    \left\|{\prod_{i=1}^n A_i}\right\|_F 
    \leq \fnorm{A_j}\prod_{i=1, i\neq j}^n \tnorm{A_i}.
\end{equation}
In fact this inequality should be true as long as the right-hand side
have one Frobenius norm component (by $\tnorm{A}\leq\fnorm{A}$).

\noindent\emph{My workaround.} Based on $\fnorm{AB}\leq
\tnorm{A}\fnorm{B}$, we can generalized this to the above equation,
therefore, I just want to make sure this is indeed correct.

\section{Paper~\cite{2022GaoMaShao_mixedprecisionJacobi}}

% 1. What did the authors try to accomplish?
The authors trying to accomplish improving the one-sided Jacobi SVD
algorithm from the LAPACK.\footnote{The computation routine
\code{xGEJSV}, which has been available since Version~3.2, implements the preconditioned Jacobi SVD algorithm
introduced by Drma\v{c} and Veseli\`{c} in
\ycite[2008]{2008DrmacVeselic_NewFastAccuratea,
2008DrmacVeselic_NewFastAccurateb}. The algorithm used by these authors
utilizes careful QR factorizations as preconditioning method.} The
authors claim that they can achieve $2\times$ speedup. To do so, they do
the following three steps:
\begin{itemize}[nosep]
    \item Preconditioning, mainly using QR factorization. 
    Given $A\in\C\nn$, first obtain $AP = Q_1R_1$ which is a
    rank-revealing QR factorization. Then if $R_1$ is diagonal dominant,
    then we are good to go. Otherwise, a LQ decomposition will be
    performed on $R_1$, $R_1 = L_1 Q_2$. Set $X = L_1$, we have $Q_1\ctp
    APQ_2\ctp = X$. 
    \item Compute the SVD in low precision. Before doing so, the authors
    check whether the low precision SVD is necessary by using the
    condition number of $R_1$, i.e. If $\kappa(R_1)$ is smaller than
    some tolerance, (usually $u_{\text{high}}^{1/4}$) then we can skip
    the low precision SVD preconditioning.  
    \item Transform the solution back to the working precision using QR
    factorization and apply the usual one-sided Jacobi to iteratively
    refine the solution. 
\end{itemize}

Moreover, they pick out some special cases that the mixed precision
Jacobi may even slower. These cases are mainly related to apply
unnecessary preconditioning methods. 

They also gives a backward analysis on their algorithm. Since the
algorithm is based on the one in
\cite{2008DrmacVeselic_NewFastAccuratea}, therefore the stability is
ensured automatically. 

% 2. What were the key elements of the approach
The key elements of this approach are the following:
\begin{itemize}[nosep]
    \item Carefully implement low precision SVD as preconditioner. 
    \item Carefully QR preconditioning as discussed in the first bullet
    point above. 
    \item The Jacobi SVD algorithm can take the advantage of almost
    orthogonal columns which leads to locally quadratic convergence.
\end{itemize}


\section{Conclusion}
These two papers are all following similar idea compare to my MSc
Project \cite{2022Zhou_mixedprecisioneigensolver}. We all use the low
precision eigen-(singular value decomposition) solver to obtain a
preconditioner. The analysis of sufficient condition for quadratic
convergence is well developed and the numerical experiments in
\cite{2022GaoMaShao_mixedprecisionJacobi,2022ZhangBai_mixedprecisionJacobi,2022Zhou_mixedprecisioneigensolver}
are all shown that the mixed precision algorithm is much more better
than the original fixed precision one. However, I think I need to trace
back to the
paper~\cite{2008DrmacVeselic_NewFastAccuratea,2008DrmacVeselic_NewFastAccurateb,
2017Drmac_Algorithm977QR, 1992DemmelVeselic_Jacobismethodis} in order to
fully understand this precondition method so that I am able to utilizes
such method in future projects. 

\emph{Some further work.} Purpose a new way of orthogonalizing the low
precision orthogonal matrix to high (double) precision. Instead of QR,
in \cite{2022Zhou_mixedprecisioneigensolver}, we implement a polar
decomposition based orthogonalization process, and this method could be
better since computing the polar decomposition using the Newton-Schulz
iteration fully utilizes the near orthogonal property of the low
precision orthogonal matrix. Moreover, we need to prove that the result
we get using the polar decomposition method is not worse than the QR
decomposition method both in computational complexity and the error.




%%%%%%%%%%%%%%%%%%%%%%%%%%% Bibliographies %%%%%%%%%%%%%%%%%%%%%%%%%%%
\newpage 
\bibliographystyle{nj-plain}
\bibliography{/Users/clement/bibtex/references.bib}


\end{document}