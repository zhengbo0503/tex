\documentclass{article}
\def\ntitle {LAPACK Eigen/SVD-Solvers}
% \def\nauthor{ } % default author is Zhengbo Zhou
% \def\notes{ } % default is the submitted version
\def\ndate{ } % default is today's date
% \def\needcrop{ }
\def\fancysec{ }
\RequirePackage{etex}
\makeatletter
\ifx \nauthor\undefined
  \def\nauthor{Zhengbo Zhou}
\else 
\fi 
\ifx \ndate\undefined 
  \def\ndate{\today}
\else 
\fi 

\author{\nauthor \thanks{%
Department of Mathematics,
University of Manchester,
Manchester, M13 9PL, England
(\texttt{zhengbo.zhou@postgrad.manchester.ac.uk}).
}}
\date{\ndate}
\title{\ntitle}

% RedeclareMathOperator
\newcommand\RedeclareMathOperator{%
  \@ifstar{\def\rmo@s{m}\rmo@redeclare}{\def\rmo@s{o}\rmo@redeclare}%
}
% this is taken from \renew@command
\newcommand\rmo@redeclare[2]{%
  \begingroup \escapechar\m@ne\xdef\@gtempa{{\string#1}}\endgroup
  \expandafter\@ifundefined\@gtempa
     {\@latex@error{\noexpand#1undefined}\@ehc}%
     \relax
  \expandafter\rmo@declmathop\rmo@s{#1}{#2}}
% This is just \@declmathop without \@ifdefinable
\newcommand\rmo@declmathop[3]{%
  \DeclareRobustCommand{#2}{\qopname\newmcodes@#1{#3}}%
}
\@onlypreamble\RedeclareMathOperator

\usepackage{algorithm}
\usepackage{algpseudocode}
\usepackage{comment}
\usepackage{bookmark}
\usepackage{microtype}
\usepackage{booktabs}
\usepackage{lastpage}
\usepackage{fancyhdr}
\usepackage{amsthm}
\usepackage{mathtools}
\usepackage{enumerate}
\usepackage{mathrsfs}
\usepackage{amsfonts}
\usepackage{amssymb}
\usepackage{tcolorbox}
\usepackage{bm}
\usepackage{cancel}
\usepackage{bbm}
\usepackage{accsupp}
\usepackage{enumitem}
\usepackage{hyperref}
\usepackage[fontsize=12pt]{fontsize}
\usepackage{geometry}
\usepackage{microtype}
\usepackage{algorithm,algorithmicx,algpseudocode}

\hbadness=99999

\geometry{
  a4paper,
  textwidth=165truemm,
  textheight=240truemm,
  top=28.5truemm,
}

\let\LaTeXStandardTableOfContents\tableofcontents
\renewcommand{\tableofcontents}{%
\begingroup%
\renewcommand{\bfseries}{\sc}%
\LaTeXStandardTableOfContents%
\endgroup%
}%
\setcounter{tocdepth}{3}
\makeatletter
\renewcommand\tableofcontents{\@starttoc{toc}}
\makeatother

\hypersetup{
  hypertexnames=false, 
  colorlinks=true,
  linkcolor=blue,
  pdfauthor={Zhengbo Zhou},
  pdftitle={\ntitle},
  pdfcreator={Zhengbo Zhou via MikTeX}
}

\usepackage[hyperpageref]{backref}
\renewcommand*{\backref}[1]{}
\renewcommand*{\backrefalt}[4]{
    \ifcase #1 %
    No citations.%
    \or
    (Cited on p.~#2.)
    \else
    (Cited on pp.~#2.)
    \fi
}



\pagestyle{fancyplain}
\fancyhead[R]{{\textit{\footnotesize\nouppercase Short Course : Function of Matrices}}}
\fancyhead[L]{\footnotesize\nouppercase\leftmark}


\newtcolorbox{mybox}[1]{colback=white!90!black,colframe=white!40!black,fonttitle=\scshape\centering,title=#1}

%%% MATLAB Code From Dr. Chris Johnson 
\usepackage{color}  
\usepackage{xcolor}
\usepackage{listings}
\definecolor{codegreen}{rgb}{0,0.6,0}
\definecolor{codegray}{rgb}{0.5,0.5,0.5}
\definecolor{codepurple}{rgb}{0.58,0,0.82}
\definecolor{mygreen}{RGB}{28,172,0}
\definecolor{mylilas}{RGB}{170,55,241}
\definecolor{backcolour}{rgb}{0.95,0.95,1.92}
\lstdefinestyle{mystyle}{
	language=matlab,
    commentstyle=\color{codegreen},
    keywordstyle=\color{blue},
    numberstyle=\footnotesize\color{codegray},
    stringstyle=\color{codepurple},
    basicstyle=\linespread{1}\ttfamily\small,
    breakatwhitespace=false,
    breaklines=true,
    captionpos=b,
    keepspaces=true,
    numbers=left,
    numbersep=10pt,
    showspaces=false,
    showstringspaces=false,
    showtabs=false,
    tabsize=4,
    aboveskip=\medskipamount,
    % frame=single,
}
\lstset{style=mystyle}
\def\inline{\lstinline[basicstyle=\upshape\ttfamily]}

%% Theorems 
\newtheorem{theorem}{Theorem}[section]
\newtheorem{proposition}[theorem]{Proposition}
\newtheorem{corollary}[theorem]{Corollary}
\newtheorem{lemma}[theorem]{Lemma}
\theoremstyle{definition}
\newtheorem{definition}[theorem]{Definition}
\newtheorem{example}[theorem]{Example}
\newtheorem{remark}[theorem]{Remark}
\newtheorem*{question}{Question}
\newtheorem*{recall}{Recall}
\newtheorem*{assumption}{Assumption}
\newtheorem*{note}{Note}
\numberwithin{equation}{section}

\let\stdsection\section
\renewcommand\section{\newpage\stdsection}

%%% Paired labels
\DeclarePairedDelimiter\ceil{\lceil}{\rceil}
\DeclarePairedDelimiter\floor{\lfloor}{\rfloor} 
\DeclarePairedDelimiter\abs{\lvert}{\rvert} % |a|
\DeclarePairedDelimiter\inner{\langle}{\rangle} % <a>
\makeatletter
\let\oldabs\abs
\def\abs{\@ifstar{\oldabs}{\oldabs*}}

%%%------------------------------------------------------------------%%%

%%% vector bold
\renewcommand{\vec}[1]{\bm{#1}}

%%%------------------------------------------------------------------%%%

%%% Real and Imaginary
\RedeclareMathOperator{\Im}{\mathrm{Im}}
\RedeclareMathOperator{\Re}{\mathrm{Re}}

%%% Integrate from ... to ...       
\newcommand{\intii}{\int_{-\infty}^{\infty}}

%%% Greek Letters
% NEVER define \l for \lambda due to Polish names in BibTeX
\renewcommand{\L}{\Lambda}
\newcommand{\vL}{\varLambda}
\newcommand{\g}{\gamma}
\newcommand{\G}{\Gamma}
\newcommand{\vG}{\varGamma}
\renewcommand{\o}{\omega}
\renewcommand{\O}{\Omega}
\newcommand{\vO}{\varOmega}
\newcommand{\s}{\sigma}
\renewcommand{\S}{\Sigma}
\newcommand{\vS}{\varSigma}
\newcommand{\eps}{\varepsilon}
\newcommand{\lap}{\varDelta}

%%% Matrix Related
\newcommand{\n}{^{n}}
\newcommand{\m}{^{m}}
\newcommand{\Rn}{\R^n}
\newcommand{\mn}{^{m\times n}}
\newcommand{\nn}{^{n\times n}}
\newcommand{\tp}{^{T}} 
\newcommand{\ctp}{^{*}}
\newcommand{\inv}{^{-1}}
\DeclareMathOperator{\diag}{diag}
\DeclareMathOperator{\rank}{rank}
\DeclareMathOperator{\tr}{trace}
\DeclareMathOperator{\range}{Range}

%%%------------------------------------------------------------------%%%

%% Norms
\newcommand{\iter}[1]{^{(#1)}} % iteration
\newcommand{\gnorm}[1]{\left\|{#1}\right\|} % general norm
\newcommand{\norm}[1]{\gnorm{#1}}
\newcommand{\tnorm}[1]{\gnorm{#1}_2} % 2-norm
\newcommand{\inorm}[1]{\gnorm{#1}_\infty} % infinity norm

%%% Over the expressions
\newcommand{\wh}{\widehat}
\newcommand{\wt}{\widetilde}
\newcommand{\wb}{\overline}

%%% Calculus
\DeclareMathOperator{\grad}{\nabla}
\renewcommand{\div}{\nabla\cdot}
\DeclareMathOperator{\curl}{\nabla\times}
\newcommand{\dd}{\mathrm{d}}
\newcommand{\pp}{\partial}
\def\eu{\mathrm{e}} % euler's constant
\def\im{\mathrm{i}} % imaginary unit

%%% Citation
\def\ycite[#1#2#3#4#5]#6{\cite[$\mit{#1#2#3#4}$#5]{#6}}

%%%------------------------------------------------------------------%%%

%%% MATHBB
\newcommand{\mb}[1]{\mathbb{#1}}
\newcommand{\N}{\mb{N}}
\newcommand{\Z}{\mb{Z}} 
\newcommand{\Q}{\mb{Q}}
\newcommand{\R}{\mb{R}} 
\newcommand{\C}{\mb{C}}
\newcommand{\F}{\mb{F}}
\renewcommand{\P}{\mb{P}} % Probability
\newcommand{\E}{\mb{E}} % Expectation
\newcommand{\V}{\mb{V}} % Variance

%%% MATHCAL and MATHSCR
\newcommand{\mc}[1]{\mathcal{#1}} % For spaces 
\newcommand{\ms}[1]{\mathscr{#1}} % For sigma-algebra
\newcommand{\mf}[1]{\mathfrak{#1}}

%%%------------------------------------------------------------------%%%

%%% Stochastic Calculus
\newcommand{\ps}{$(\Omega,\ms F,\P)$}
\newcommand{\ito}{It\^o}
\DeclareMathVersion{bold}
\newcommand{\indi}{\mathbbm{1}} % indicator function
\providecommand*{\napprox}{%
  \BeginAccSupp{method=hex,unicode,ActualText=2249}%
  \not\approx
  \EndAccSupp{}%
}
\mathchardef\gang="2D
\DeclareMathOperator*{\esssup}{esssup}

\DeclareMathOperator{\law}{\mathfrak{Law}}
\newcommand{\loc}{\textup{loc}}
\newcommand{\locmart}{\mc M_{\loc}^C}
\newcommand{\locm}{\mc M_{\loc,T}^{C,0}}
\newcommand{\locmt}{\wt{\mc M}_{\loc,T}^{C,0}}
\newcommand{\nicee}[1]{\mathcal{E}_{#1}^H(B)}


\DeclareMathOperator{\sign}{diag}

\makeatother

\renewcommand{\emph}[1]{\textit{\color{purple} #1}}
\renewcommand{\bf}[1]{\textsf{\bfseries \color{purple} #1}}

\newcommand{\code}[1]{\texttt{\color{green!30!black} #1}}

\begin{document}
\maketitle
% \tableofcontents
\thispagestyle{firstpage} 


\begin{abstract}
    This note concentrate on
    paper~\cite{gms22,zhba22}. 
\end{abstract}

%%%%%%%%%%%%%%%%%%%%%%%%%%%% MAIN ARTICLE %%%%%%%%%%%%%%%%%%%%%%%%%%%%
\section{Introduction}
LAPACK provides several solvers for symmetric eigenvalue problems(SYP). 
However, the Jacobi algorithm is not explicitly used in the LAPACK 
routines for SYP. 
Several papers using LAPACK's singular value decomposition(SVD) routines
to help computing the eigenvalues. 

This note is based on the following two papers
\cite{gms22} and \cite{zhba22}.

\begin{itemize}[nosep]
    \item Weiguo Gao, Yuxin Ma, and Meiyue Shao.
    \href{https://arxiv.org/abs/2209.04626}{A mixed precision
    {Jacobi} {SVD} algorithm}. {\em ArXiv}, 2022.
    \item Zhiyuan Zhang and Zheng-Jian Bai.
    \href{https://arxiv.org/abs/2211.03339}{A mixed precision {Jacobi}
      method for the symmetric eigenvalue problem}.{\em ArXiv}, 2022.
  \end{itemize}

\subsection{LAPACK Name Scheme}
All drives and computational routines have names of the form 
\bf{XYYZZZ}, where for some driver routine the 6th character is blank.

The first letter, \bf{X}, indicates the data type as follows:
\begin{equation}\notag
    \begin{aligned}
        \text{S} \quad & \text{Real} \\
        \text{D} \quad & \text{Double Precision} \\ 
        \text{C} \quad & \text{Complex} \\ 
        \text{Z} \quad & \text{Double Complex}
    \end{aligned}
\end{equation}
When we wish to refer to an LAPACK routine generically, regardless of
data type, we replace the first letter by ``x''. Thus xGESV refers to
any or all of the routines \code{SGESV}, \code{CGESV}, \code{DGESV} and
\code{ZGESV}.

The next two letters, \bf{YY}, indicate the type of matrix (or of the
most significant matrix). Most of these two-letter codes apply to both
read and complex matrices; a few apply specifically to one or the other,
as indicated in Table~\ref{tab.LNS}.

\begin{table}[ht]
    \centering
    \caption{Matrix types in the LAPACK naming scheme}
    \label{tab.LNS}
    \begin{tabular}{c p{0.9\linewidth}}
        \toprule
        BD & \bf{B}i\bf{D}iagonal\\
        DI & \bf{DI}agonal\\
        GB & \bf{G}eneral \bf{B}and \\ 
        GE & \bf{GE}neral (i.e. unsymmetric, in some cases rectangular)
        \\
        GG & \bf{G}eneral matrices, \bf{G}eneralized problem (i.e. a
        pair of general matrices) \\
        GT & \bf{G}eneral \bf{T}ridiagonal \\
        HB & (complex) \bf{H}ermitian \bf{B}and\\
        HE & (complex) \bf{HE}rmitian\\
        HG & upper \bf{H}essenberg matrix, \bf{G}eneralized problem
        (i.e. a Hessenberg and a triangular matrix)\\
        HP & (complex) \bf{H}ermitian, \bf{P}acked storage\\
        HS & upper \bf{H}e\bf{S}senberg\\
        OP & (real) \bf{O}rthogonal, \bf{P}acked storage\\
        OR & (real) \bf{OR}thogonal \\
        PB & symmetric or Hermitian \bf{P}ositive definite \bf{B}and \\
        PO & symmetric or Hermitian \bf{PO}sitive definite \\
        PP & symmetric or Hermitian \bf{P}ositive definite, \bf{P}acked storage \\
        PT & symmetric or Hermitian \bf{P}ositive definite \bf{T}ridiagonal \\
        SB & (real) \bf{S}ymmetric \bf{B}and \\
        SP & \bf{S}ymmetric, {P}acked storage \\
        ST & (real) \bf{S}ymmetric \bf{T}ridiagonal \\
        SY & \bf{SY}mmetric \\
        TB & \bf{T}riangular {B}and \\
        TG & \bf{T}riangular matrices, \bf{G}eneralized problem (i.e., a pair of triangular matrices)\\
        TP & \bf{T}riangular, \bf{P}acked storage\\
        TR & \bf{TR}iangular (or in some cases quasi-triangular)\\
        TZ & \bf{T}rape\bf{Z}oidal\\
        UN & (complex) \bf{UN}itary\\
        UP & (complex) \bf{U}nitary, \bf{P}acked storage\\
        \bottomrule
    \end{tabular}
\end{table}
When we wish to refer to a class of routines that performs the same
function on different types of matrices, we replace the first three
letters by ``xyy''. Thus \code{xyySVX} refers to all the expert driver
routines for systems of linear equations that are listed in
Table~\ref{tab.LNS}.

The last three letters \bf{ZZZ} indicate the computation performed. They
are explained in \ycite[1999, Sec~2.4]{abbb99_LUG}.
For example, \code{SGEBRD} is a \bf{S}ingle precision routine that
performs a bidiagonal reduction (\bf{BRD}) of a real \bf{GE}neral
matrix. 

%%%%%%%%%%%%%%%%%%%%%%%%%%%%%%%%%%%%%%%%%%%%%%%%%%%%%%%%%%%%%%%%%%%%%%%%
\newpage
\section{LAPACK Routine for SVD Problem by \cite{gms22}}

The paper~\cite{gms22} is working with the SVD problem rather
the SEP. However, they point out the following LAPACK routines that they
used for their implementation of the Jacobi SVD. We will list these
routines and give some brief introductions to them.

The following 4 routines are used in \cite{gms22} for computing
SVD. 
\begin{itemize}[nosep]
    \item[Section \ref{sec.DGEJSV}] \code{S/D/C/Z GE JSV}
    \item[Section \ref{sec.DGESVJ}] \code{S/D/C/Z GE SVJ}
    \item[Section \ref{sec.DGEQRF}] \code{S/D/C/Z GE QRF}
    \item[Section \ref{sec.SGESVD}] \code{S/D/C/Z GE SVD}
\end{itemize}

The following 2 routines are used in \cite{gms22} for computing
eigendecomposition. 
\begin{itemize}[nosep]
    \item[Section \ref{sec.DSYEV}] \code{S/D SY EV}
\end{itemize}

%%%%%%%%%%%%%%%%%%%%%%%%%%%%%%%%%%%%%%%%%%%%%%%%%%%%%%%%%%%%%%%%%%%%%%%%
\subsection[\code{DGEJSV}]{\code{DGEJSV} : Double Precision, General
Matrices, Preconditioned Jacobi SVD Algorithm}
\label{sec.DGEJSV}
\code{DGEJSV} : Computes the \emph{singular value
decomposition} of a matrix $A \in \R\mn$, where $m \ge n$.
\code{DGEJSV} can sometimes compute tiny singular values and their
singular vectors much more accurately than other SVD routines.
\code{DGEJSV} implements a \emph{preconditioned Jacobi SVD
algorithm}. It uses \code{DGEQP3}, \code{DGEQRF}, and \code{DGELQF}
as a preprocessor, which in some cases results in much higher
accuracy.  

\begin{example}
    Suppose matrix $A$ has the structure $A = D_1CD_2$, where $D_1$ and
    $D_2$ are arbitrarily ill-conditioned diagonal matrices and $C$ is
    well-conditioned matrix. In this case, complete pivoting in the
    first QR factorizations provides accuracy dependent on the condition
    number of $C$, and independent of $D_1$ and $D_2$. 
\end{example}

\begin{example}
    If $A$ can be written as $A = BD$, with well-conditioned $B$ and
    some diagonal $D$, then the high accuracy is guaranteed, both
    theoretically and in software, independent of $D$. 
\end{example}
For more details, see \cite{drve08i,drve08ii}

%%%%%%%%%%%%%%%%%%%%%%%%%%%%%%%%%%%%%%%%%%%%%%%%%%%%%%%%%%%%%%%%%%%%%%%%
\subsection[\code{DGESVJ}]{\code{DGESVJ} : Double Precision, General
Matrices, Computing SVD using Jacobi Plane Rotations}
\label{sec.DGESVJ}
\code{DGESVJ} computes the SVD of a matrix $A \in \R\mn$, where $M \ge
N$. The SVD of $A$ is written as 
\begin{equation}\notag
    A = U \Sigma V \tp.
\end{equation}
\code{DGESVJ} can sometimes compute tiny singular values and their
singular vectors much more accurately than other SVD routine. 

The orthogonal $n \times n$ matrix $V$ is obtained as a product of
Jacobi plane rotations. The rotations are implements as fast scaled
rotations of Anda and Park~\cite{anpa94}. In case of underflow of
the Jacobi angle, a modified Jacobi transformation of
Drma\u{c}~\cite{drma97} is used. The relative accuracy of
the computed singular values and the accuracy of the computed singular
vectors (in angle metric) is as guaranteed by the theory of Demmel and
Veselic~\cite{deve92}. The condition number that determines the
accuracy in the full rank case is essentially $\min_{\Delta\text{ is
diagonal}}\kappa_2(A\Delta)$. The best performance of this Jacobi SVD
procedure is achieved if used in an accelerated version of Drma\u{c} and
Veseli\'{c}~\cite{drve08i,drve08ii}.

%%%%%%%%%%%%%%%%%%%%%%%%%%%%%%%%%%%%%%%%%%%%%%%%%%%%%%%%%%%%%%%%%%%%%%%%
\subsection[\code{DGEQRF}]{\code{DGEQRF} : Double Precision, General
Matrices, QR Factorization}
\label{sec.DGEQRF}
\code{DGEQRF} : Computes a QR factorization of a real $m\times n$ matrix
$A$: 
\begin{equation}\notag
    A = Q \times 
    \begin{bmatrix}
        R \\ 0
    \end{bmatrix}
\end{equation} 
where $Q \in \R^{m\times m}$ is orthogonal, $R \in \R\nn$ is an upper
triangular matrix, and $0$ is a $(m-n)\times n$ zero matrix if $m > n$.
The matrix $Q$ is represented as a product of elementary reflectors 
\begin{equation}\notag
    Q = H_1 H_2 \cdots H_k, \quad k = \min(m,n).
\end{equation}
Each $H_i$ has the form 
\begin{equation}\notag
    H_i = I - \tau vv\tp 
\end{equation}
where $\tau$ is a real scalar, and $v$ is a real vector with $v(1:i-1) =
0$ and $v(i) = 1$.

%%%%%%%%%%%%%%%%%%%%%%%%%%%%%%%%%%%%%%%%%%%%%%%%%%%%%%%%%%%%%%%%%%%%%%%%

\subsection[\code{DGESVD}]{\code{DGESVD} : Single Precision, General
Matrices, Singular Value Decomposition}
\label{sec.SGESVD}
\code{DGESVD} computes the singular value decomposition (SVD) of a real
$m\times n$ matrix $A$. The SVD is written 
\begin{equation}\notag
    A = U \Sigma V\tp
\end{equation}
where $\Sigma \in \R\mn$ is an matrix which is zero except for its
$\min(m,n)$ diagonal elements, and $U\in\R^{m\times m}, V \in \R\nn$ are
orthogonal matrices. The diagonal elements of $\Sigma$ are the singular
values of $A$; they are real and non-negative, and are returned in
descending oreder. The first $\min(m,n)$ columns of $U$ and $V$ are the
left and right singular vectors of $A$.

\code{xGESVD}, where $\code{x} \in \code{S,C}$, computes the SVD of a
general matrix by 
\begin{itemize}[nosep]
    \item Reducing it to bidiagonal form $B$ via routine
    \code{xGEBRD}\footnote{\code{xGEBRD} reduce a general $m\times n$
    matrix $A$ to 
    upper or lower bidiagonal form $B$ by an orthogonal transformation:
    $Q\tp AP = B$.}.
    \item Call \code{xBDSQR}\footnote{\code{xBDSQR} computes the
    singular values of a 
    real $n\times n$ bidiagonal matrix $B$ using the implicit zero-shift
    QR algorithm.} to compute the SVD of $B$.
\end{itemize}
%%%%%%%%%%%%%%%%%%%%%%%%%%%%%%%%%%%%%%%%%%%%%%%%%%%%%%%%%%%%%%%%%%%%%%%%
\subsection[\code{DSYEV}]{\code{DSYEV} : Double Precision, Symmetric
Matrices, Eigendecomposition}
\label{sec.DSYEV}
\code{DSYEV} : A \emph{simple} driver computes all the eigenvalues and
(optionally) eigenvectors by 
\begin{itemize}[nosep]
    \item Call \code{SSYTRD} to reduce symmetric matrix to tridiagonal
    form. 
    \item For eigenvalues only, call
    \code{SSTERF}\footnote{\code{SSTERF} computes all
    eigenvalues of a symmetric tridiagonal matrix using the
    Pal-Walker-Kahan variant of the QL or QR algorithm.}. 
    For eigenvectors, first call \code{SORGTR}\footnote{\code{SORGTR}
    generates a real orthogonal matrix $Q$ which is defined as the
    product of $n-1$ elementary reflectors of order $N$.} to generate
    the orthogonal matrix, then call
    \code{SSTEQR}\footnote{\code{SSTEQR} computes all eigenvalues and,
    optionally, eigenvectors of a symmetric tridiagonal matrix using the
    implicit QL or QR method.  The eigenvectors of a full or band
    symmetric matrix can also be found if \code{SSYTRD} has been used
    to reduce this matrix to  tridiagonal form.}.
\end{itemize}

%%%%%%%%%%%%%%%%%%%%%%%%%%%%%%%%%%%%%%%%%%%%%%%%%%%%%%%%%%%%%%%%%%%%%%%%

\begin{remark}
    Paper~\ycite[2022]{zhba22} uses \code{SSYEV} to compute the
    eigenvalues and eigenvectors of a given matrix in low precision.
\end{remark}


%%%%%%%%%%%%%%%%%%%%%%%%%%%% Bibliographies %%%%%%%%%%%%%%%%%%%%%%%%%%%%

\bibliographystyle{/Users/clement/Dropbox/bibtex/nj-plain}
\bibliography{/Users/clement/Dropbox/bibtex/strings.bib,
  /Users/clement/Dropbox/bibtex/ref.bib} 


\end{document}

%%% Local Variables:
%%% mode: latex
%%% TeX-master: t
%%% TeX-master: t
%%% TeX-master: t
%%% TeX-master: t
%%% TeX-master: t
%%% End:
