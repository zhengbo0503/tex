\documentclass{article}
\def\ntitle {Sylvester Equation Solver}
% \def\nauthor{ } % default author is Zhengbo Zhou
% \def\notes{ } % default is the submitted version
% \def\ndate{ } % default is today's date
% \def\needcrop{ } % crop for easy previewing
% \def\fancysec{ } % section font become fancier 
\RequirePackage{etex}
\makeatletter
\ifx \nauthor\undefined
  \def\nauthor{Zhengbo Zhou}
\else 
\fi 
\ifx \ndate\undefined 
  \def\ndate{\today}
\else 
\fi 

\author{\nauthor \thanks{%
Department of Mathematics,
University of Manchester,
Manchester, M13 9PL, England
(\texttt{zhengbo.zhou@postgrad.manchester.ac.uk}).
}}
\date{\ndate}
\title{\ntitle}

% RedeclareMathOperator
\newcommand\RedeclareMathOperator{%
  \@ifstar{\def\rmo@s{m}\rmo@redeclare}{\def\rmo@s{o}\rmo@redeclare}%
}
% this is taken from \renew@command
\newcommand\rmo@redeclare[2]{%
  \begingroup \escapechar\m@ne\xdef\@gtempa{{\string#1}}\endgroup
  \expandafter\@ifundefined\@gtempa
     {\@latex@error{\noexpand#1undefined}\@ehc}%
     \relax
  \expandafter\rmo@declmathop\rmo@s{#1}{#2}}
% This is just \@declmathop without \@ifdefinable
\newcommand\rmo@declmathop[3]{%
  \DeclareRobustCommand{#2}{\qopname\newmcodes@#1{#3}}%
}
\@onlypreamble\RedeclareMathOperator

\usepackage{algorithm}
\usepackage{algpseudocode}
\usepackage{comment}
\usepackage{bookmark}
\usepackage{microtype}
\usepackage{booktabs}
\usepackage{lastpage}
\usepackage{fancyhdr}
\usepackage{amsthm}
\usepackage{mathtools}
\usepackage{enumerate}
\usepackage{mathrsfs}
\usepackage{amsfonts}
\usepackage{amssymb}
\usepackage{tcolorbox}
\usepackage{bm}
\usepackage{cancel}
\usepackage{bbm}
\usepackage{accsupp}
\usepackage{enumitem}
\usepackage{hyperref}
\usepackage[fontsize=12pt]{fontsize}
\usepackage{geometry}
\usepackage{microtype}
\usepackage{algorithm,algorithmicx,algpseudocode}

\hbadness=99999

\geometry{
  a4paper,
  textwidth=165truemm,
  textheight=240truemm,
  top=28.5truemm,
}

\let\LaTeXStandardTableOfContents\tableofcontents
\renewcommand{\tableofcontents}{%
\begingroup%
\renewcommand{\bfseries}{\sc}%
\LaTeXStandardTableOfContents%
\endgroup%
}%
\setcounter{tocdepth}{3}
\makeatletter
\renewcommand\tableofcontents{\@starttoc{toc}}
\makeatother

\hypersetup{
  hypertexnames=false, 
  colorlinks=true,
  linkcolor=blue,
  pdfauthor={Zhengbo Zhou},
  pdftitle={\ntitle},
  pdfcreator={Zhengbo Zhou via MikTeX}
}

\usepackage[hyperpageref]{backref}
\renewcommand*{\backref}[1]{}
\renewcommand*{\backrefalt}[4]{
    \ifcase #1 %
    No citations.%
    \or
    (Cited on p.~#2.)
    \else
    (Cited on pp.~#2.)
    \fi
}



\pagestyle{fancyplain}
\fancyhead[R]{{\textit{\footnotesize\nouppercase Short Course : Function of Matrices}}}
\fancyhead[L]{\footnotesize\nouppercase\leftmark}


\newtcolorbox{mybox}[1]{colback=white!90!black,colframe=white!40!black,fonttitle=\scshape\centering,title=#1}

%%% MATLAB Code From Dr. Chris Johnson 
\usepackage{color}  
\usepackage{xcolor}
\usepackage{listings}
\definecolor{codegreen}{rgb}{0,0.6,0}
\definecolor{codegray}{rgb}{0.5,0.5,0.5}
\definecolor{codepurple}{rgb}{0.58,0,0.82}
\definecolor{mygreen}{RGB}{28,172,0}
\definecolor{mylilas}{RGB}{170,55,241}
\definecolor{backcolour}{rgb}{0.95,0.95,1.92}
\lstdefinestyle{mystyle}{
	language=matlab,
    commentstyle=\color{codegreen},
    keywordstyle=\color{blue},
    numberstyle=\footnotesize\color{codegray},
    stringstyle=\color{codepurple},
    basicstyle=\linespread{1}\ttfamily\small,
    breakatwhitespace=false,
    breaklines=true,
    captionpos=b,
    keepspaces=true,
    numbers=left,
    numbersep=10pt,
    showspaces=false,
    showstringspaces=false,
    showtabs=false,
    tabsize=4,
    aboveskip=\medskipamount,
    % frame=single,
}
\lstset{style=mystyle}
\def\inline{\lstinline[basicstyle=\upshape\ttfamily]}

%% Theorems 
\newtheorem{theorem}{Theorem}[section]
\newtheorem{proposition}[theorem]{Proposition}
\newtheorem{corollary}[theorem]{Corollary}
\newtheorem{lemma}[theorem]{Lemma}
\theoremstyle{definition}
\newtheorem{definition}[theorem]{Definition}
\newtheorem{example}[theorem]{Example}
\newtheorem{remark}[theorem]{Remark}
\newtheorem*{question}{Question}
\newtheorem*{recall}{Recall}
\newtheorem*{assumption}{Assumption}
\newtheorem*{note}{Note}
\numberwithin{equation}{section}

\let\stdsection\section
\renewcommand\section{\newpage\stdsection}

%%% Paired labels
\DeclarePairedDelimiter\ceil{\lceil}{\rceil}
\DeclarePairedDelimiter\floor{\lfloor}{\rfloor} 
\DeclarePairedDelimiter\abs{\lvert}{\rvert} % |a|
\DeclarePairedDelimiter\inner{\langle}{\rangle} % <a>
\makeatletter
\let\oldabs\abs
\def\abs{\@ifstar{\oldabs}{\oldabs*}}

%%%------------------------------------------------------------------%%%

%%% vector bold
\renewcommand{\vec}[1]{\bm{#1}}

%%%------------------------------------------------------------------%%%

%%% Real and Imaginary
\RedeclareMathOperator{\Im}{\mathrm{Im}}
\RedeclareMathOperator{\Re}{\mathrm{Re}}

%%% Integrate from ... to ...       
\newcommand{\intii}{\int_{-\infty}^{\infty}}

%%% Greek Letters
% NEVER define \l for \lambda due to Polish names in BibTeX
\renewcommand{\L}{\Lambda}
\newcommand{\vL}{\varLambda}
\newcommand{\g}{\gamma}
\newcommand{\G}{\Gamma}
\newcommand{\vG}{\varGamma}
\renewcommand{\o}{\omega}
\renewcommand{\O}{\Omega}
\newcommand{\vO}{\varOmega}
\newcommand{\s}{\sigma}
\renewcommand{\S}{\Sigma}
\newcommand{\vS}{\varSigma}
\newcommand{\eps}{\varepsilon}
\newcommand{\lap}{\varDelta}

%%% Matrix Related
\newcommand{\n}{^{n}}
\newcommand{\m}{^{m}}
\newcommand{\Rn}{\R^n}
\newcommand{\mn}{^{m\times n}}
\newcommand{\nn}{^{n\times n}}
\newcommand{\tp}{^{T}} 
\newcommand{\ctp}{^{*}}
\newcommand{\inv}{^{-1}}
\DeclareMathOperator{\diag}{diag}
\DeclareMathOperator{\rank}{rank}
\DeclareMathOperator{\tr}{trace}
\DeclareMathOperator{\range}{Range}

%%%------------------------------------------------------------------%%%

%% Norms
\newcommand{\iter}[1]{^{(#1)}} % iteration
\newcommand{\gnorm}[1]{\left\|{#1}\right\|} % general norm
\newcommand{\norm}[1]{\gnorm{#1}}
\newcommand{\tnorm}[1]{\gnorm{#1}_2} % 2-norm
\newcommand{\inorm}[1]{\gnorm{#1}_\infty} % infinity norm

%%% Over the expressions
\newcommand{\wh}{\widehat}
\newcommand{\wt}{\widetilde}
\newcommand{\wb}{\overline}

%%% Calculus
\DeclareMathOperator{\grad}{\nabla}
\renewcommand{\div}{\nabla\cdot}
\DeclareMathOperator{\curl}{\nabla\times}
\newcommand{\dd}{\mathrm{d}}
\newcommand{\pp}{\partial}
\def\eu{\mathrm{e}} % euler's constant
\def\im{\mathrm{i}} % imaginary unit

%%% Citation
\def\ycite[#1#2#3#4#5]#6{\cite[$\mit{#1#2#3#4}$#5]{#6}}

%%%------------------------------------------------------------------%%%

%%% MATHBB
\newcommand{\mb}[1]{\mathbb{#1}}
\newcommand{\N}{\mb{N}}
\newcommand{\Z}{\mb{Z}} 
\newcommand{\Q}{\mb{Q}}
\newcommand{\R}{\mb{R}} 
\newcommand{\C}{\mb{C}}
\newcommand{\F}{\mb{F}}
\renewcommand{\P}{\mb{P}} % Probability
\newcommand{\E}{\mb{E}} % Expectation
\newcommand{\V}{\mb{V}} % Variance

%%% MATHCAL and MATHSCR
\newcommand{\mc}[1]{\mathcal{#1}} % For spaces 
\newcommand{\ms}[1]{\mathscr{#1}} % For sigma-algebra
\newcommand{\mf}[1]{\mathfrak{#1}}

%%%------------------------------------------------------------------%%%

%%% Stochastic Calculus
\newcommand{\ps}{$(\Omega,\ms F,\P)$}
\newcommand{\ito}{It\^o}
\DeclareMathVersion{bold}
\newcommand{\indi}{\mathbbm{1}} % indicator function
\providecommand*{\napprox}{%
  \BeginAccSupp{method=hex,unicode,ActualText=2249}%
  \not\approx
  \EndAccSupp{}%
}
\mathchardef\gang="2D
\DeclareMathOperator*{\esssup}{esssup}

\DeclareMathOperator{\law}{\mathfrak{Law}}
\newcommand{\loc}{\textup{loc}}
\newcommand{\locmart}{\mc M_{\loc}^C}
\newcommand{\locm}{\mc M_{\loc,T}^{C,0}}
\newcommand{\locmt}{\wt{\mc M}_{\loc,T}^{C,0}}
\newcommand{\nicee}[1]{\mathcal{E}_{#1}^H(B)}


\DeclareMathOperator{\sign}{diag}

\makeatother

%% redefine \emph and \bf such that they are colored and can be easily
%% transformed back to normal \emph and \bf
\renewcommand{\emph}[1]{\textit{\color{purple} #1}}
\renewcommand{\bf}[1]{\textsf{\bfseries \color{purple} #1}}

\begin{document}
\maketitle
\tableofcontents
\thispagestyle{firstpage}
%%%%%%%%%%%%%%%%%%%%%%%%%%%%%%%%%%%%%%%%%%%%%%%%%%%%%%%%%%%%%%%%%%%%%%
%%%%%%%%%%%%%%%%%%%%%%%%%%%% MAIN ARTICLE %%%%%%%%%%%%%%%%%%%%%%%%%%%%
%%%%%%%%%%%%%%%%%%%%%%%%%%%%%%%%%%%%%%%%%%%%%%%%%%%%%%%%%%%%%%%%%%%%%%

\section{Introduction}
\label{sec.introduction}

Let $A\in\R^{m\times m}$ and $B\in\R\nn$ be given matrices and
define the linear transformation: $\phi(X) :\R\mn \to \R\mn$ by 
\begin{equation}\notag
  \phi(X) = AX + XB.
\end{equation}
The linear transformation is nonsingular if and only if $A$ and $-B$ have
no eigenvalues in common.
This has been formalized by Bhatia and Rosenthal in \cite{bhro97}
and we will discussed in the Section \ref{sec.solv-sylv-equat}.

Linear equations of the form
\begin{equation}\label{eq.sylv-eqn}
  \phi(X) = AX + XB = C
\end{equation}
is called the Sylvester equation,
which first studied by James Joseph Sylvester in 1884.
In his paper, he considered the homogeneous case $AX - XB = 0$.

\subsection{Solvablility of the Sylvester Equation}
\label{sec.solv-sylv-equat}
Let us first discuss a simple case when $B = -A$.
The Sylvester equation \eqref{eq.sylv-eqn} becomes
\begin{equation}\label{eq.simple-sylv}
  AX - XA = C,
\end{equation}
and taking the trace of both sides gives
\begin{equation}\notag
  \tr(C) = \tr(AX - XA) = \tr(AX) - \tr(XA) = 0. 
\end{equation}
Hence for the simplified Sylvester equation~\eqref{eq.simple-sylv},
a solution can exist only when $C$ has zero trace.
For example, $AX - XA = I$ has no solution. 

For the matrix case, Sylvester proved it in his original paper.
Moreover, Rosenblum \cite{rose69} provided a operator case later.

\begin{theorem}[Sylvester-Rosenblum Theorem]
\label{thm.sylv-rose}
If $A$ and $B$ are operators such that
$\sigma(A)\cup \sigma(-B) = \emptyset$,
then the equation $AX + XB = Y$ has a unique solution $X$ for
every operator $Y$.
\end{theorem}

\begin{proof}
The proof can be seen from the later section when we discuss the
Bartel-Stewart algorithm.
\end{proof}

\section{The Bartels-Stewart Algorithm}
In 1972, Bartels and Stewart proposed an algorithm that computes the
solution to the Sylvester equation~\ycite[1972]{bast72}.
This method has enjoyed considerable success \ycite[1976]{bemc76}.
The crux of the Bartels-Stewart algorithm is to use the QR algorithm
to compute the real Schur decomposition
\begin{equation}\label{eq.bsalg.decomp}
  A = U R U\tp, \quad B = V S V\tp,
\end{equation}
where $R,S$ are upper quasi-triangular and $U$ and $V$ are orthogonal.
Then, we premultiplying the Sylvester equation $AX + XB = C$ by $U\tp$
and postmultiplying by $V$, we have
\begin{equation}\label{eq.upper-tri-sylv}
  U\tp C V = U\tp A X V + U\tp X B V = U\tp A UU\tp X V + U\tp X VV\tp B V.
\end{equation}
Let $F\coloneqq U\tp CV$ and $Y = U\tp XV$, the Sylvester equation becomes
\begin{equation}\notag
  F = RY + YS, \qquad \text{$R,S$ are quasi-upper triangular.}
\end{equation}
Consider $Y = [y_{1}|\cdots|y_{n}]$ and $F = [f_{1}|\cdots|f_{n}]$,
the \eqref{eq.upper-tri-sylv} can be decompose to $n$ upper triangular
linear system, where the $j$th columns on both sides leads to
\begin{equation}\label{eq.sylv-linear-sys}
  (R + s_{jj}I)y_{j} = f_{j} - \sum_{k=1}^{j-1}s_{kj}y_{k},
  \qquad j = 1\colon n.
\end{equation}
This is a upper triangular linear system which can be solved by backward
substitution efficiently. From \eqref{eq.sylv-linear-sys}, $y_{j}$ is
uniquely determined if and only if $R + s_{jj}I$ is not singular, which
means $\Lambda(R)\cup \Lambda(-S) = \emptyset$ and this proves Theorem
\ref{thm.sylv-rose}. Finally, if $Y$ can be uniquely determined, then $X$
can also be uniquely determined via $X = UYV\tp$.

\section{The Hessenberg-Schur Algorithm}
\label{sec.hess-schur-algor}

In this section, we provide another algorithm described by Golub, Nash and
Van Loan~\cite{gnv79} which, instead of, computing the Schur decomposition,
it uses an upper Hessenberg matrix instead. Namely, it modified
\eqref{eq.bsalg.decomp} into
\begin{align*}
  H = & U \tp A U, \qquad \text{$H$ is upper Hessenberg,} \\
  S = & V \tp B V, \qquad \text{$S$ is upper triangular.}
\end{align*}
Recall that a matrix $H = (h_{ij})$ is upper Hessenberg if $h_{ij} = 0$ for
all $i > j + 1$. The orthogonal reduction of $A$ to upper Hessenberg form
can be accomplished by Householder matrices in $10m^{3}/3$
flops~\ycite[2013, Alg.~7.4.2]{gova13_mc4}, and this is a significant
reduction in computational expenses. For general matrix $A$, a Schur
decomposition will typically requires $10n^{3}$ flops according to
\ycite[1979]{gnv79}. Then, following the exactly same procedure, we can
transform the Sylvester equation \eqref{eq.sylv-eqn} into
\begin{equation}\notag
  HY + YS = F, \quad (H + s_{jj}I)y_{j} = f_{j} -
  \sum_{k=1}^{j-1}s_{kj}y_{k},
  \qquad j = 1\colon n. 
\end{equation}
The above system can be solved using Gaussian elimination with partial
pivoting, and it only requires $O(m^{2})$  flops after the right-hand side
has been computed. 







%%%%%%%%%%%%%%%%%%%%%%%%%%%%%%%%%%%%%%%%%%%%%%%%%%%%%%%%%%%%%%%%%%%%%%
%%%%%%%%%%%%%%%%%%%%%%%%%%% Bibliographies %%%%%%%%%%%%%%%%%%%%%%%%%%%
%%%%%%%%%%%%%%%%%%%%%%%%%%%%%%%%%%%%%%%%%%%%%%%%%%%%%%%%%%%%%%%%%%%%%%
\newpage
\bibliographystyle{/Users/clement/Dropbox/bibtex/nj-plain}
\bibliography{/Users/clement/Dropbox/bibtex/strings.bib,
  /Users/clement/Dropbox/bibtex/ref.bib} 
%% APPENDICES



\end{document}


%%% Local Variables:
%%% mode: latex
%%% TeX-master: t
%%% End:
