\chapter{Supplementary Explanation}
\section{Algorithm~\ref{alg:jacobi-pair}}
\label{app:algorithm-jacobi-pair}
We have two different expressions for computing $t$ as shown in Table~\ref{tab:different-approaches-for-t}.
\begin{table}[ht]
  \centering 
  \caption{Two different approaches for computing $t$ in Algorithm~\ref{alg:jacobi-pair}.}
  \label{tab:different-approaches-for-t}
  \begin{tabular}{ccc}
    \toprule 
    & Approach 1 & Approach 2 \\ \midrule  
    $\tau \geq 0$ & $ t = - \tau + \sqrt{\tau^2 + 1}$ & $\displaystyle t = \frac{1}{\tau + \sqrt{1 + \tau^2}}$ \\ \midrule 
    $\tau < 0 $ &$ t = - \tau - \sqrt{\tau^2 + 1}$ & 
    $\displaystyle t = \frac{1}{\tau - \sqrt{1 + \tau^2}}$\\
    \bottomrule
  \end{tabular}
\end{table}

These two expressions are the same by simple calculation,
\begin{equation}
  t=\frac{1}{\tau+\sqrt{1+\tau^{2}}}=\frac{\tau-\sqrt{1+\tau^{2}}}{\left(\tau+\sqrt{1+\tau^{2}}\right)\left(\tau-\sqrt{1+\tau^{2}}\right)}=-\tau+\sqrt{1+\tau^{2}}.
\end{equation}

However, when we implement these into MATLAB, approach 1 will output $c = 1, s = 0$ for $a_{pq} \approx 10^{-8}$, and approach 2's output gives $s \neq 0$ even for $a_{pq} \approx 10^{-14}$. Namely, by using approach 1, the computed output is less accurate than the output from approach 2.

\chapter{Supplementary Code}

\section{Testing Matrices}\label{app:myrandsvd}
By using the following code, we can generate a random real symmetric matrix with predefined condition number.
\begin{lstlisting}
function A = my_randsvd(n, kappa, mode)
classname = 'double';
if isempty(mode)
  mode = 'Geo';
end
switch mode % Set up vector sigma of singular values.
  case {'Geometric','geometric','Geo','geo'} 
    % geometrically distributed singular values
    factor = kappa^(-1/(n-1));
    sigma = factor.^(0:n-1);
  case {'Arithmatic', 'arithmatic', 'Ari', 'ari'}
    % arithmetically distributed singular values
    sigma = ones(n,1) - (0:n-1)'/(n-1)*(1-1/kappa);
  otherwise
    error(message('MATLAB:randsvd:invalidMode'));
end
sigma = cast(sigma,classname); signs = sign(rand(1,n-2)); 
sigma(2:n-1) = sigma(2:n-1).* signs; 
Q = qmult(n,1,classname);
A = Q*diag(sigma)*Q';
A = (A + A')/2;
\end{lstlisting}

\section{Full Jacobi Algorithm Code}\label{app:code-full-jacobi}
\begin{lstlisting}
function [V,D,counter] = jacobi_eig(Aini,tol,type,maxiter)
%JACOBI_EIG   Jacobi Eigenvalue Algorithm
%   [A,V,counter] = JACOBI_EIG(Aini,tol,type,maxiter) computes the eigen-
%   decomposition of Aini, Aini = VDV', where V is orthogonal.
%Input:
%   Aini -------- Input matrix
%   tol --------- Desired tolerance, usually 1.1e-16 (double precision)
%   type -------- Type of Jacobi algorithm, either
%                   * 'classical': For classical Jacobi algorithm
%                   * 'cyclic': For cyclic-by-row Jacobi algorithm
%   maxiter ----- Maximum number of sweep, specially for cyclic-by-row
%               Jacobi algorithm. If undetermined, the maximum sweep will
%               be set as 10 sweeps.
%Output:
%   V ------------ Orthogonal matrix
%   D ------------ Resulting matrix after apply Jacobi algorithm.
%   counter ------ Number of iteration
if nargin <= 2, error('Not enough input arguments.');
elseif nargin >= 5, error('Too many input arguments.');
else
  % select method, 'classical' or 'cyclic'
  switch type
    case {'classical','Classical'}
      method = 1;
    case {'cyclic','Cyclic'}
      method = 2;
    otherwise
      error("The Jacobi algorithm avaliable are " + ...
        "'classical' and 'cyclic'")
  end
  switch method
    case 1
      [V,D,counter] = jacobi_classical(Aini,tol);
    case 2
      switch nargin
        case 4
          maxiteration = maxiter;
          [V,D,counter] = jacobi_cyclic(Aini,tol,maxiteration);
        case 3
          maxiteration = 10;
          [V,D,counter] = jacobi_cyclic(Aini,tol,maxiteration);
        otherwise
          error('Not enough input arguments.');
      end
  end
end
end

% classical Jacobi algorithm
function [V,A,counter] = jacobi_classical(A,tol)
counter = 0; n = length(A); V = eye(n); done_rot = true;
tol1 = tol * norm(A);
while done_rot
  if isint(counter/(n*n)), A = (A + A')/2; end
  done_rot = false; [p,q] = maxoff(A);
  if abs(A(p,q)) >  tol1 * sqrt(abs(A(p,p) * A(q,q)))
    counter = counter + 1; done_rot = true;
    [c,s] = jacobi_pair(A,p,q);
    J = [c,s;-s,c]; 
    A([p,q],:) = J'*A([p,q],:);
    A(:,[p,q]) = A(:,[p,q]) * J;
    V(:,[p,q]) = V(:,[p,q]) * J;
  else
    A = diag(diag(A));
    break;
  end
end
end

% Cyclic-by-row Jacobi algorithm
function [V,A,iter] = jacobi_cyclic(A,tol,maxiter)
n = length(A); V = eye(n); iter = 0; done_rot = true;
while done_rot && iter < maxiter
  done_rot = false;
  for p = 1:n-1
    for q = p+1:n
      if abs(A(p,q)) > tol * sqrt(abs(A(p,p)*A(q,q)))
        done_rot = true;
        [c,s] = jacobi_pair(A,p,q);
        J = [c,s;-s,c];
        A([p,q],:) = J'*A([p,q],:);
        A(:,[p,q]) = A(:,[p,q]) * J;
        V(:,[p,q]) = V(:,[p,q]) * J;
      end
    end
  end
  if done_rot 
    A = (A + A')/2; iter = iter + 1; 
  else
    A = diag(diag(A));
    return;
  end
%   fprintf('off(A) = %d, sweep = %d\n', off(A), iter);
end
end

% compute the Frobenius norm of the off-diagonal entries of the input matrix
function num = off(A)
n = length(A); A(1:n+1:n*n) = 0;
num = norm(A,'fro');
end

% find the index of maximum off-diagonal entry of the input matrix A
function [p,q] = maxoff(A)
n = length(A); % dimension of matrix A
A(1:n+1:n*n) = 0; A = abs(A); % clear the diagonal entries
[val, idx1] = max(A);
[~, q] = max(val); p = idx1(q);
end

% calculate the Givens rotation matrix's entries (c,s).
function [c,s] = jacobi_pair(A,p,q)
if A(p,q) == 0
  c = 1; s = 0;
else
  tau = (A(q,q)-A(p,p))/(2*A(p,q));
  if tau >= 0
    t = 1/(tau + sqrt(1+tau*tau));
  else
    t = 1/(tau - sqrt(1+tau*tau));
  end
  c = 1/sqrt(1+t*t);
  s = t*c;
end
end
\end{lstlisting}

\section{Code for Figure~\ref{fig:classical-cyclic-compare}}\label{app:code-for-fig1}
Figure~\ref{fig:classical-cyclic-compare} can be regenerated using the following routine 
\begin{lstlisting}
clc; clear; close all; format short e;
condi1 = 100; tol = 2^(-53);
rng(1,'twister');
for n = 1e2:1e2:1e3
  A = my_randsvd(n,condi1,'geo');
  fprintf('iteration of n %d/%d\n', n/1e2, 1e3/1e2);
  tic 
  [V1,D1,iter1] = jacobi_eig(A,tol,'Classical');
  t_classical_n(n/1e2) = toc;
  tic
  [V2,D2,iter2] = jacobi_eig(A,tol,'Cyclic',20);
  t_cyclic_n(n/1e2) = toc;
end
nplot = 1e2:1e2:1e3;
n = 2.5e2; 
rng(1,'twister');
for condi = 1e3:1e3:1e5
  A = my_randsvd(n,condi,'geo');
  fprintf('iteration of cond %d/%d...',condi/1e3,1e5/1e3);
  tic
  [V1,D1,iter1] = jacobi_eig(A,tol,'Classical');
  t_classical_condi(condi/1e3) = toc;
  tic
  [V2,D2,iter2] = jacobi_eig(A,tol,'Cyclic',20);
  t_cyclic_condi(condi/1e3) = toc;
  fprintf('complete \n');
end
condiplot = 1e3:1e3:1e5;
close all;
subplot(1,2,1); hold off;
plot(nplot,t_classical_n,'-^','LineWidth',2);
hold on;
plot(nplot,t_cyclic_n,'-^','LineWidth',2);
xlabel('Dimension $(n)$');
ylabel('Time Used (sec)');
title('Fixed $\kappa(A) = 100$');
legend('Classical Jacobi','Cyclic Jacobi','Location','northwest','FontSize',20);
axis square
subplot(1,2,2); hold off;
plot(condiplot,t_classical_condi,'-^','LineWidth',1.5);
hold on;
plot(condiplot,t_cyclic_condi,'-^','LineWidth',1.5);
xlabel('$\kappa(A)$');
ylabel('Time Used (sec)');
title('Fixed $n = 250$');
legend('Classical Jacobi','Cyclic Jacobi','Location','northwest','FontSize',20);
axis square; axis([0,1e5,0,25])
\end{lstlisting}

\section{Code for Figure~\ref{fig:orthog}}\label{app:fig:orthog}
Figure~\ref{fig:orthog} can be regenerated using the following routine 
\begin{lstlisting}
clc; clear; close all; format short e; 
ud = 2^(-53); n = 1e2:1e2:3e3; condi = 100; rng(1,'twister');
for i = 1:length(n)
  fprintf('iteration %d/%d\n', i, length(n));
  A = gallery('randsvd', n(i), -condi, 3);
  [Q,~] = myqr(A); 
  orthog(i) = norm(Q'*Q - eye(n(i)));
end
semilogy(n,orthog,'-^r',LineWidth=1.5); 
hold on; semilogy(n, n.*ud, '-^k',LineWidth=1.5);
legend("$\|Q^TQ - I_n\|$", '$nu_d$','Location','northwest');
xlabel('Dimension $(n)$')
\end{lstlisting}

\section{Code for Figure~\ref{fig:seconditer}}\label{app:seconditer}
Figure~\ref{fig:seconditer} can be regenerated using the following routine 
\begin{lstlisting}
clc; clear; close all;
n = 10:10:8e4;
sq = (sqrt(1+n .* eps(single(1/2))) - 1).^4;
sq_ref = n .* eps(1/2);
plot(n,sq); hold on;
plot(n,sq_ref,'--');
plot(52400,5.81757e-12,'^k')
legend('$\|X_2 - U\|_2$','$n\cdot u_d$','Location','northwest');
xlabel('Dimension $(n)$')
\end{lstlisting}

\section{Code for Figure~\ref{fig:nsorthog}}\label{app:fig:nsorthog}
Figure~\ref{fig:nsorthog} can be regenerated using the following routine 
\begin{lstlisting}
clc; clear; close all; format short e; rng(1,'twister');
ud = 2^(-53); 
n = 1e2:1e2:3e3; condi = 1e2;
for i = 1:length(n)
  fprintf('Start iteration %d/%d,\n',i,length(n));
  A = my_randsvd( n(i), condi, 'geo');
  [Q,~] = eig(single(A)); Q = double(Q);
  Q1 = newton_schulz(Q); 
  orthog(i) = norm(Q1'*Q1 - eye(n(i)));
end
semilogy(n,orthog,'-^r',LineWidth=1.5); 
hold on; 
semilogy(n, n.*ud, '-^k',LineWidth=1.5);
legend("$\|Q_d^*Q_d - I_n\|$",'$nu_d$','Location','northwest'); 
xlabel('Dimension (n)'); title('Fix $\kappa(A) = 100$'); axis square
\end{lstlisting}

\section{Code for Figure~\ref{fig:acccompare}}\label{app:fig:acccompare}
Figure~\ref{fig:acccompare} can be regenerated using the following routine 
\begin{lstlisting}
clc; clear; close all; format short e; rng(1,'twister');
ud = 2^(-53); n = 1e2:1e2:3e3; condi = 1e2;
for i = 1:length(n)
  fprintf('iteration %d/%d\n',i,length(n));
  A = my_randsvd(n(i),condi,'geo');  
  [Q,~] = eig(single(A)); Q = double(Q);
  Q1 = newton_schulz(Q); 
  [Q2,~] = qr(Q);
  [Q3,~] = myqr(Q);
  orthogns(i) = norm(Q1'*Q1 - eye(n(i)));
  orthogqr(i) = norm(Q2'*Q2 - eye(n(i)));
  orthogmyqr(i) = norm(Q3'*Q3 - eye(n(i)));
end
semilogy(n,orthogns,'-^',LineWidth=1.5); hold on; 
semilogy(n,orthogqr,'-^',LineWidth=1.5); 
semilogy(n,orthogmyqr,'-^',LineWidth=1.5); 
legend("Newton-Schulz", ...
  "\texttt{qr()} from MATLAB", ...
  "My own QR code",...
  'interpreter','latex', ...
  'Location','northwest', ...
  'FontSize',20); 
xlabel('Dimension $(n)$');
ylabel('$\|Q_d^*Q_d - I_n\|$')
axis([0,3000,0,2e-14])
axis square
\end{lstlisting}


\section{Code for Figure~\ref{fig:timecompare}}
\label{app:fig:timecompare}
Figure~\ref{fig:timecompare} can be regenerated using the following routine 
\begin{lstlisting}
clc; clear; close all; format short e; 
ud = 2^(-53); n = 1e2:1e2:3e3; condi = 100;
time_ns = zeros(length(n),1); time_qr = time_ns;
rng(1,'twister'); % RNG
for i = 1:length(n)
  fprintf('iteration %d/%d\n',i,length(n));
  A = my_randsvd(n(i),condi,'geo');
  [Q,~] = eig(single(A)); Q = double(Q);
  tic; [Q1] = newton_schulz(Q); time_ns(i) = toc;
  tic; [Q2,~] = myqr(Q); time_qr(i) = toc;
end
semilogy(n,time_ns,'-^',LineWidth=1.5); hold on; 
semilogy(n,time_qr,'-^',LineWidth=1.5); 
legend('Newton--Schulz','QR Factorization','Location','northwest','interpreter','latex','FontSize',20); 
xlabel('Dimension $(n)$')
ylabel('Time (sec)')
axis square  
\end{lstlisting}


\section{Mixed Precision Jacobi Algorithm Code}\label{app:mixed-precision-jacobi}
\begin{lstlisting}
function [Q,D,iter] = jacobi_precondi(A,tol,maxiter)
%JACOBI_PRECONDI(A,tol,maxiter) compute the eigendecomposition of A using cyclic-by-row Jacobi algorithm with preconditioning.
%% Preliminaries
switch nargin
  case 2, max_it = 10; case 3, max_it = maxiter;
  otherwise, error('Please check the number of inputs');
end
[n,m] = size(A); 
if m ~= n, error('Input matrix should be square matrix'), end
if issymmetric(A) == 0, error('Input matrix should be symmetric'); end
%% Eigendecomposition at single precision
[Q_low,~] = eig(single(A)); Q_low = double(Q_low);
%% Newton--Schulz iteration to orthogonalize
Q_d = newton_schulz(Q_low);
%% Cyclic-by-row Jacobi algorithm on preconditioned A
A_cond = Q_d' * A * Q_d;
[V,D,iter] = jacobi_eig(A_cond,tol,'Cyclic',max_it);
%% Output
Q = Q_d * V;
\end{lstlisting}



\section{Code for Figure~\ref{fig:precondition-accuracy} }\label{app:precondition-accuracy}
Figure~\ref{fig:precondition-accuracy} can be regenerated using the following routine 
\begin{lstlisting}
clc; clear; format short e;
n = 50:50:1e3; condi = 500; nplot = n;
rng(1,'twister');
for i = 1:length(n)
  fprintf('iteration %d/%d...',i,length(n))
  A_geo = my_randsvd(n(i),condi,'geo');
  A_ari = my_randsvd(n(i),condi,'ari');
  [V1,D1,iter1] = jacobi_precondi(A_geo,eps(1/2));
  [V2,D2,iter2] = jacobi_precondi(A_ari,eps(1/2));
  accuracy_n_geo(i) = norm(A_geo * V1 - V1 * D1);
  accuracy_n_ari(i) = norm(A_ari * V2 - V2 * D2);
  reference_n(i) = n(i) * norm(A_geo) * eps(1/2);
  fprintf('complete\n');
end
%%
subplot(1,2,1); hold off;
semilogy(nplot,accuracy_n_geo,'-^'); hold on;
semilogy(nplot,accuracy_n_ari,'-^');
semilogy(nplot,nplot.*eps(1/2),'--k');
xlabel('Dimension $(n)$'); 
ylabel('$\|AQ - Q\Lambda\|_2/\|A\|_2$');
legend('Geometric','Arithmetic','$nu_d\|A\|_2$','Location','southeast','FontSize',20);
title('Fixed $\kappa(A) = 500$')
axis square
%%
condiplot = 1e2:1e2:1e4; n = 5e2;
for i = 1:length(condiplot)
  fprintf('iteration %d/%d...',i,length(condiplot))
  A_geo = my_randsvd(n,condiplot(i),'geo');
  A_ari = my_randsvd(n,condiplot(i),'ari');
  [V1,D1,iter1] = jacobi_precondi(A_geo,eps(1/2));
  [V2,D2,iter2] = jacobi_precondi(A_ari,eps(1/2));
  accuracy_condi_geo(i) = norm(A_geo * V1 - V1 * D1);
  accuracy_condi_ari(i) = norm(A_ari * V2 - V2 * D2);
  reference_n(i) = n * norm(A_geo) * eps(1/2);
  fprintf('complete\n');
end
%%
close all;
subplot(1,2,1); hold off;
semilogy(nplot,accuracy_n_geo,'-^'); hold on;
semilogy(nplot,accuracy_n_ari,'-^');
semilogy(nplot,nplot.*eps(1/2),'--k');
xlabel('Dimension $(n)$'); 
ylabel('$\|AQ - Q\Lambda\|_2/\|A\|_2$');
legend('Geometric','Arithmetic','$nu_d$','Location','southeast','FontSize',20);
title('Fixed $\kappa(A) = 500$')
axis square
subplot(1,2,2); hold off;
semilogy(condiplot,accuracy_condi_geo,'-^'); hold on;
semilogy(condiplot,accuracy_condi_ari,'-^');
semilogy(condiplot,reference_n,'--k'); 
xlabel('$\kappa(A)$'); 
ylabel('$\|AQ - Q\Lambda\|_2/\|A\|_2$');
legend('Geometric','Arithmetic','$nu_d$','Location','northeast','FontSize',20);
title('Fixed $n = 500$')
axis square; axis([0,1e4,0,8e-14]);
\end{lstlisting}

\section{Code for Figure~\ref{p3:fig:approx-eig-test}}
\label{p3:sec:figure1}
\begin{lstlisting}
clc; clear all; close all; figure(1);
n = [50,250,500];
for i = 1:3
  subplot(2,2,i)
  testing_1(n(i));
  if i == 1
    axis([0,1e4,0,6e-6]);
    title('Fixed $n=50$');
  elseif i == 2
    axis([0,1e4,0,3.8e-5]);
    title('Fixed $n = 250$');
  elseif i == 3
    axis([0,1e4,0,8e-5]);
    title('Fixed $n = 500$');
  end
  pause(0.1);
end
subplot(2,2,4); hold off
condi = 1e2; n1 = 10:10:1e3; offA = zeros(1,length(n1)); rng(1,'twister');
for i = 1:length(n1)
  fprintf('start %d/%d\n',i,length(n1));
  A_geo = my_randsvd(n1(i),condi,'geo');
  A_ari = my_randsvd(n1(i),condi,'ari');
  [Q1,~] = eig(single(A_geo)); Q1 = double(Q1);
  Q1_d = newton_schulz(Q1);
  offA_geo(i) = off(Q1_d' * A_geo * Q1_d)/norm(A_geo);
  [Q2,~] = eig(single(A_ari)); Q2 = double(Q2);
  Q2_d = newton_schulz(Q2);
  offA_ari(i) = off(Q2_d' * A_ari * Q2_d)/norm(A_ari);
end
semilogy(n1,offA_geo); hold on;
semilogy(n1,offA_ari);
semilogy(n1,n1 .* double(eps(single(1/2))),'--k');
legend('Geometric','Arithmetic','$nu_\ell$','Location','southeast');
xlabel('$\kappa(A)$'); ylabel('off$(A_{cond})/\|A\|_2$','Interpreter','latex');
set(gca,'FontSize',20);
axis square;
title('Fixed $\kappa(A) = 100$')
function testing_1(n)
condi = 10:10:1e4; offA_geo = zeros(1,length(condi)); offA_ari = offA_geo;
rng(1,'twister');
for i = 1:length(condi)
  fprintf('start %d/%d\n',i,length(condi));
  A_geo = my_randsvd(n,condi(i),'geo');
  [Q1,~] = eig(single(A_geo)); Q1 = double(Q1);
  Q1_d = newton_schulz(Q1);
  offA_geo(i) = off(Q1_d' * A_geo * Q1_d)/norm(A_geo);
  A_ari = my_randsvd(n,condi(i),'ari');
  [Q2,~] = eig(single(A_ari)); Q2 = double(Q2);
  Q2_d = newton_schulz(Q2);
  offA_ari(i) = off(Q2_d' * A_ari * Q2_d)/norm(A_ari);
end
semilogy(condi,offA_geo,'LineWidth',1.5); hold on;
semilogy(condi,offA_ari);
semilogy(condi,n * double(eps(single(1/2))) * ones(1,length(condi)),'--k','LineWidth',1.5);
legend('Geometric','Arithmetic','$nu_\ell$');
xlabel('$\kappa(A)$'); ylabel('off$(A_{cond})/\|A\|_2$','Interpreter','latex');
set(gca,'FontSize',20);
axis square;
end
\end{lstlisting}

\section{Code for Figure~\ref{fig:typical-sweep}}\label{app:typical-sweep}
Figure~\ref{fig:typical-sweep} can be regenerated using the following routine
\begin{lstlisting}
clear; clc; rng(1,'twister');
n = 50:50:1e3; condi = 500; nplot = n;
for i = 1:length(n)
  A_geo = my_randsvd(n(i),condi,'geo');
  [V1,D1,iter1] = jacobi_precondi(A_geo,eps(1/2));
  iteration_n_geo(i) = iter1;
  A_ari = my_randsvd(n(i),condi,'ari');
  [V2,D2,iter2] = jacobi_precondi(A_ari,eps(1/2));
  iteration_n_ari(i) = iter2;
  fprintf('iteration %d/%d complete\n',i,length(n));
end
condiplot = 1e2:1e2:1e4; n = 5e2;
for i = 1:length(condiplot)
  A_geo = my_randsvd(n,condiplot(i),'geo');
  [V1,D1,iter1] = jacobi_precondi(A_geo,eps(1/2));
  iteration_condi_geo(i) = iter1;
  A_ari = my_randsvd(n,condiplot(i),'ari');
  [V2,D2,iter2] = jacobi_precondi(A_ari,eps(1/2));
  iteration_condi_ari(i) = iter2;
  fprintf('iteration %d/%d complete\n',i,length(condiplot));
end
close all
subplot(1,2,1);
semilogy(nplot,iteration_n_geo,'-^'); hold on;
semilogy(nplot,iteration_n_ari,'-^');
xlabel('Dimension $(n)$');
ylabel('Number of iterations til converge');
title('Fixed $\kappa(A) = 500$')
legend('Geometric','Arithmetic','Location','northeast');
axis square; axis([0,1000,0,4]);
subplot(1,2,2);
semilogy(condiplot,iteration_condi_geo,'-^'); hold on;
semilogy(condiplot,iteration_condi_ari,'-^');
legend('Geometric','Arithmetic','Location','northeast');
xlabel('$\kappa(A)$'); 
ylabel('Number of iterations til converge');
title('Fixed $n = 500$')
axis square; axis([0,1e4,0,5]);
\end{lstlisting}

\section{Code for Figure~\ref{fig:time-improvement}}\label{app:time-improvement}
Figure~\ref{fig:time-improvement} can be regenerated using the following routine
\begin{lstlisting}
clc; clear; close all; rng(1,'twister');
n = 1e2:1e2:5e2; condi = 500; tol = 2^(-53);
for i = 1:length(n)
  Ag = my_randsvd(n(i),condi,'geo');
  Aa = my_randsvd(n(i),condi,'ari');
  tic; [V1,D1,iter1] = jacobi_precondi(Ag,tol); 
  t_precond_g(i) = toc;
  tic; [V2,D2,iter2] = jacobi_eig(Ag,tol,'Cyclic'); 
  t_normal_g(i) = toc;
  tic; [Va1,Da1,iter1] = jacobi_precondi(Aa,tol); 
  t_precond_a(i) = toc;
  tic; [Va2,Da2,iter2] = jacobi_eig(Aa,tol,'Cyclic'); 
  t_normal_a(i) = toc;
  fprintf('Iteration %d/%d complete\n',i,length(n));
end
subplot(1,2,1)
plot(n,t_normal_g,'-^');
hold on;
plot(n,t_precond_g,'-^');
xlabel('Dimension $(n)$')
ylabel('Time Used (sec)')
legend('Without Preconditioning', 'With Preconditioning','Location','northwest');
title('Geometrically distributed test matrices')
axis square  
subplot(1,2,2)
plot(n,t_normal_a,'-^');
hold on;
plot(n,t_precond_a,'-^');
xlabel('Dimension $(n)$')
ylabel('Time Used (sec)')
legend('Without Preconditioning', 'With Preconditioning','Location','northwest');
title('Arithmetically distributed test matrices')
axis square  
\end{lstlisting}