\chapter{Conclusion and Further Work}

We studied the classical Jacobi algorithm and the cyclic-by-row Jacobi algorithm and concluded that the cyclic-by-row Jacobi algorithm is much more efficient than the other one. Then, after performing implementation and testing on the QR factorization and the Newton--Schulz iteration, in order to orthogonalize an almost orthogonal matrix, we shall always use the Newton--Schulz iteration. Finally, we followed the preconditioning technique in~\ycite[2000]{DRMAC2000-preconditioner} and successfully propose the mixed precision algorithm~\ref{alg:jacobi-preconditioned} for the real symmetric eigenproblem. By utilizing the \inline{eig} function at single precision and preconditioning, we can save roughly 75\% of the time compared to applying the Jacobi algorithm alone.

Further works of this thesis include:
\begin{enumerate}
  \item Rounding error analysis hasn't been conducted.
  \item There exist more ways to orthogonalize an almost orthogonal matrix. For example, the Cholesky QR factorization.
  \item Only one low precision level has been studied. These theories can apply to half precision as well. However, it requires more work on exploring a half precision version of \inline{eig} function in order to become faster than our Algorithm~\ref{alg:jacobi-preconditioned} which uses MATLAB built-in single precision. 
  \item There are other ways that can be used to speed up the Jacobi algorithm. For example, the threshold Jacobi algorithm~\ycite[1965, Section~5.12]{wilk65} and the block Jacobi procedure~\ycite[2013, Section~8.5.6]{van2013mc}.
\end{enumerate}