\usepackage[english]{babel}
\usepackage{bookmark,booktabs}
\usepackage{fancyhdr}
\fancyhf{}
\fancyhead[RE]{\nouppercase{\small \leftmark}}
\fancyhead[LO]{\nouppercase{\small Section \rightmark}}
\fancyhead[LE,RO]{\small\thepage}
\renewcommand{\headrulewidth}{0pt}
\renewcommand{\footrulewidth}{0pt}
\setlength{\headheight}{15pt}

%%% Algorithm %%%
\usepackage{algorithm}
\usepackage{algpseudocode}
%%%
\usepackage{color,bold-extra,mathrsfs,float,bbm}
\usepackage[tableposition=t]{caption}
\captionsetup{textfont={small,it},labelfont={small, sc}}
\captionsetup[figure]{name=Fig.}
\usepackage{comment,graphics,aliascnt}
\usepackage[dvipsnames]{xcolor}
\hypersetup{
  colorlinks=true,
  linkcolor=red,
  urlcolor=red,
  citecolor=blue,
  bookmarksopen=true,
}
\usepackage[hyperpageref]{backref}
\renewcommand*{\backref}[1]{}% for backref < 1.33 necessary
\renewcommand*{\backrefalt}[4]{%
\ifcase #1 %
No citations.%
\or
(Cited on page #2.)
\else
(Cited on pages #2.)
\fi
}
\usepackage{enumerate,amsmath,amsfonts,amssymb,comment,mathtools}
\usepackage{amsthm}
% Fonts %%

% \usepackage[
%     scale = 0.9,
% ]{lucimatx}
% % mathbb for lucimatx
% \AtBeginDocument{
%   \DeclareSymbolFont{AMSb}{U}{msb}{m}{n}
%   \DeclareSymbolFontAlphabet{\mathbb}{AMSb}}
% \renewcommand{\familydefault}{ppl}

\usepackage[T1]{fontenc}
\DeclareFontFamily{OT1}{rsfs}{}
\DeclareFontShape{OT1}{rsfs}{n}{it}{<-> rsfs10}{}
\DeclareMathAlphabet{\mathscr}{OT1}{rsfs}{n}{it}
\usepackage{lmodern} \normalfont %to load T1lmr.fd
\DeclareFontShape{T1}{lmr}{bx}{sc} { <-> ssub * cmr/bx/sc }{}
\usefont{T1}{qzc}{m}{it}
\usepackage{eucal}

% \pdfmapfile{=mtpro2.map}
% \newcommand\hmmax{0}
% \newcommand\bmmax{0}
% \usepackage{txfonts}
% \usepackage[T1]{fontenc}
% \usepackage[LY1]{fontenc}
% \usepackage[utf8]{inputenc}
% \usepackage{textcomp}
% \usepackage[lite,slantedGreek,eufrak,eucal,zswash]{mtpro2}
% \let\mathbb=\varmathbb

%% MATLAB Code %%% From Dr. Chris Johnson
\usepackage{listings}
\definecolor{codegreen}{rgb}{0,0.6,0}
\definecolor{codegray}{rgb}{0.5,0.5,0.5}
\definecolor{codepurple}{rgb}{0.58,0,0.82}
\definecolor{mygreen}{RGB}{28,172,0}
\definecolor{mylilas}{RGB}{170,55,241}
\definecolor{backcolour}{rgb}{0.95,0.95,1.92}
\lstdefinestyle{mystyle}{
	language=matlab,
  commentstyle=\color{codegreen},
  keywordstyle=\color{blue},
  numberstyle=\tiny\color{codegray},
  stringstyle=\color{codepurple},
  basicstyle=\linespread{1}\ttfamily\footnotesize,
  breakatwhitespace=false,
  breaklines=true,
  captionpos=b,
  keepspaces=true,
  numbers=left,
  numbersep=5pt,
  showspaces=false,
  showstringspaces=false,
  showtabs=false,
  tabsize=2,
  aboveskip=\medskipamount,
  frame=single,
}
\lstset{style=mystyle}
% \inline is a custom macro
\def\inline{\lstinline[basicstyle=\upshape\ttfamily]}





\DeclareCaptionLabelSeparator{separation}{:\quad}

%% Some definitions %%
\DeclarePairedDelimiter\ceil{\lceil}{\rceil}
\DeclarePairedDelimiter\floor{\lfloor}{\rfloor}
\usepackage{bm}
\DeclarePairedDelimiter\abs{\lvert}{\rvert}
\DeclarePairedDelimiter\inner{\langle}{\rangle}
% real and imaginary
\renewcommand{\Re}[1]{\mathfrak{Re}\left\{{#1}\right\}}
\renewcommand{\Im}[1]{\mathfrak{Im}\left\{{#1}\right\}}

\newcommand{\intii}{\int_{-\infty}^{\infty}}


%% Colored Box %%
\usepackage[most]{tcolorbox}
\tcbset{colback=green!40!white, arc=0pt,outer arc=0pt, boxrule=0pt, frame empty, breakable,
        highlight math style= {enhanced, %<-- needed for the ’remember’ options
            colframe=red,colback=red!10!white,boxsep=0pt}
        }
\newtcbox{\mybox}[1][red]
  {on line, arc = 0pt, outer arc = 0pt,
    colback = #1!10!white, colframe = #1!50!black,
    boxsep = 0pt, left = 1pt, right = 1pt, top = 2pt, bottom = 2pt,
    boxrule = 1pt, bottomrule = 1pt, toprule = 1pt}

%% Theorems %%
\newtheorem{theorem}{Theorem}[chapter]
\newtheorem{lemma}[theorem]{Lemma}
\newtheorem{proposition}[theorem]{Proposition}
\newtheorem{assumption}[theorem]{Assumption}
\newtheorem{corollary}[theorem]{Corollary}

\theoremstyle{definition}
\newtheorem{definition}[theorem]{Definition}
\newtheorem{remark}[theorem]{Remark}
\newtheorem{example}[theorem]{Example}
\newtheorem*{note}{Note}
\newtheorem*{question}{Question}
\newtheorem*{recall}{Recall}
\newtheorem{exercise}[theorem]{Exercise}

\numberwithin{equation}{chapter}

%% mathbb %%
\newcommand{\N}{\mathbb{N}} % natural numbers 
\newcommand{\Z}{\mathbb{Z}} % integers 
\newcommand{\Q}{\mathbb{Q}} % rationals
\newcommand{\R}{\mathbb{R}} % reals
\newcommand{\C}{\mathbb{C}} % complex

%% Probability %%
\renewcommand{\P}{\mathbb{P}} % probability
\DeclareMathOperator{\E}{\mathbb{E}} % expectation
\DeclareMathOperator{\V}{\mathbf{Var}} % variance

%% matrix and vectors %%
\newcommand{\n}{^n}
\newcommand{\Rn}{\R^n}
\newcommand{\mn}{^{m\times n}}
\newcommand{\nn}{^{n\times n}}
\renewcommand{\ij}{_{ij}}
\newcommand{\tp}{^T} 
\newcommand{\ctp}{^*}
\newcommand{\inv}{^{-1}}
\newcommand{\diag}{\mathrm{diag}}
\newcommand{\rank}{\mathrm{rank}}
\newcommand{\tr}{\mathrm{trace}}
\renewcommand{\det}{\mathrm{det}}
\newcommand{\range}{Range}
\renewcommand{\dim}{dim}
\renewcommand{\span}{span}
\newcommand{\eref}[1]{\eqref{#1}}

%% Symbols %%
\newcommand{\mat}{MATLAB}
\newcommand{\wh}{\widehat}
\newcommand{\wt}{\widetilde}
\newcommand{\wb}{\overline}
\newcommand{\grad}{\nabla}
\renewcommand{\div}{\nabla\cdot}
\newcommand{\curl}{\nabla\times}
\newcommand{\lap}{\varDelta}
\newcommand{\dd}{\mathrm{d}}
\newcommand{\pp}{\partial}
\def\eu{\mathrm{e}} % euler's constant
\def\im{\mathrm{i}} % imaginary unit

%%%% Greek Letters %%%%
\renewcommand{\a}{\alpha}
\renewcommand{\l}{\lambda}
\renewcommand{\L}{\Lambda}
\newcommand{\vL}{\varLambda}
\renewcommand{\b}{\beta}
\newcommand{\p}{\phi}
\newcommand{\x}{\xi}
\newcommand{\G}{\Gamma}
\newcommand{\vG}{\varGamma}
\renewcommand{\O}{\Omega}
\newcommand{\vO}{\varOmega}
\renewcommand{\o}{\omega}
\newcommand{\e}{\epsilon}

% MSc project macros%
\def\ycite[#1#2#3#4#5]#6{\cite[$\mit{#1#2#3#4}$#5]{#6}}
\newcommand{\iter}[1]{^{(#1)}} % iteration
\newcommand{\norm}[1]{\|{#1}\|_2} % 2-norm
\newcommand{\sign}[1]{\mathrm{sign}\left({#1}\right)}
\renewcommand{\Sigma}{\varSigma} % skew sigma
\newcommand{\off}{\mathsf{off}} % off operator
